%
%  $Description: Author guidelines and sample document in LaTeX 2.09/2e$
%
%  $Author: Priit Ruberg$
%  $Date: 2015/02/09 $
%  $Revision: 2.5 $
%
%  Translated to English 2017/03/10 by Chris Raastad
%
\documentclass[12pt]{article} %Document class definition and text size settings
%
%Packages can be explored in more details: https://www.ctan.org/pkg/PACKAGE_NAME?lang=en
%
\usepackage{graphicx} %Allow using graphics in the text
\usepackage[top=2.5cm, bottom=2.5cm, left=3cm, right=3cm]{geometry} %Set the page margins
\usepackage{titlesec} %Package for title style
\usepackage{longtable} %Package so tables can be longer than one page
\usepackage{multirow} %Package so table cells can span multiple rows
\usepackage[colorinlistoftodos]{todonotes} %Package so you can add nice TODO marks in your paper with \todo{TODO text...}
\usepackage{cite} %For Bibtex

\usepackage{hyperref}
% \usepackage{url} %Package in order to nicely use URLs
\usepackage{float} %Package to improves interface for defining floating objects like figures and tables

\usepackage[english, estonian]{babel} %Specifies possible languages of the document: English, Russian, and Estonian
	\addto\captionsestonian{\def\refname{\centerline{References}}} %Changes references name and makes it center
	\addto\captionsestonian{\def\listfigurename{\centerline{List of figures}}} %Changes drawing list name and makes it center
	\addto\captionsestonian{\def\listtablename{\centerline{List of tables}}} %Changes table list name makes it center
	\addto\captionsestonian{\def\contentsname{\centerline{Table of contents}}}
\usepackage[T2A,T1]{fontenc} %Font encodings for Russian and Estonian letters
\usepackage[utf8]{inputenc} %use UTF8 decodings

\usepackage{tocloft} %Control table of contents, tables, etc.
%\setlength\cftparskip{-2pt}
%\setlength\cftbeforechapskip{0pt}

\usepackage{amssymb} %For square itemized listss
\renewcommand{\labelitemi}{\tiny$\blacksquare$} %For square itemized lists


\usepackage{caption} %Needed to customise captions for tables and figures
\captionsetup{labelsep=period} %Set table and figure caption name to be separated with text with a period

\usepackage{verbatimbox} %To put program code in the center using Verbatim

\titlelabel{\thetitle.\quad} %Adds periods to the end of titles

\usepackage{times} %Sets font to Times New Roman
\usepackage{fancyhdr} %Allows more control of headers and footers
\setlength{\parindent}{0cm} %Set paragraph indentation to zero
\usepackage{setspace} %Allows setting spacing between lines
\onehalfspacing %Set spacing to 1.5x
\usepackage{parskip}
\setlength{\parskip}{\baselineskip}
%\hangindent=0.7cm

\hyphenation{põhi-tekstis üliõpilas-kood lehe-küljed joonda-takse} %Correcting incorrect hyphenation (?)

\usepackage{eurosym}
\usepackage{quoting} %for quoting format
\quotingsetup{vskip=0pt} %for no vertical whitespace around quotings

% a bunch of stuff to make paragraph act as a subsubsubsection and numbered and formatted correctly
\usepackage{titlesec}
\setcounter{secnumdepth}{4}
\titleformat{\paragraph}
{\normalfont\normalsize\bfseries}{\theparagraph}{1em}{}
\titlespacing*{\paragraph}
{0pt}{3.25ex plus 1ex minus .2ex}{1.5ex plus .2ex}

% --- new commands ---

% creates hyperref using the name of a section
\newcommand{\hypernameref}[1]{\hyperref[#1]{\nameref{#1}}}

% creates a hyperref in form "Table #" to a table
\newcommand{\hypertableref}[1]{\hyperref[#1]{Table \ref{#1}}}

% make font look like code
\def\code#1{\texttt{#1}}

% ability to add line breaks in tables (by making a new table...)
% http://tex.stackexchange.com/questions/2441/how-to-add-a-forced-line-break-inside-a-table-cell
\newcommand{\specialcell}[2][c]{%
  \begin{tabular}[#1]{@{}l@{}}#2\end{tabular}}

\begin{document}
\selectlanguage{english}

%------------------------------ENGLISH TITLE PAGE---------------------------------
\thispagestyle{fancy} %Page will include header and footer
\renewcommand{\headrulewidth}{0pt} %Remove header horizontal line
\renewcommand{\footrulewidth}{0pt} %Remove footer horizontal line
\headheight = 57pt %Set header heght (with regards to compiler suggestion)
\footskip = 11pt %Footer space
\headsep = 0pt %Decrease header and text line spacing distance to zero

\chead{ %Place the following text header to the center
 \textsc{\begin{Large} %Make the following text have big letters
	TALLINN UNIVERSITY OF TECHNOLOGY\\
	\end{Large}}
	Faculty of Information Technology\\
	Department of Computer Engineering
}
\vspace*{7 cm} %Make the page beginning and text line spacing correspond to the width

\begin{center} %Text centered
ITC70LT\\[0cm]
Christopher David Raastad\\
\begin{LARGE}
\textsc{Security, Scalability, and Privacy Analysis of the Euro 2.0 Digital Currency\\}
\end{LARGE}
Master thesis\\[2cm]
\end{center}

\begin{flushright} %Align text to the right
Alex Norta\\[0cm]
PhD\\[0cm]
Associated Professor\\[0cm]
\end{flushright}

\cfoot{Tallinn 2017} %Add location and year to the header
%\renewcommand{\headrulewidth}{0pt} %Remove the footer horizontal line
\pagebreak %End of page

%------------------------------TIITELLEHT EESTI KEELES---------------------------------
\selectlanguage{estonian}
\thispagestyle{fancy} %Leht sisaldab päist ja jalust
\renewcommand{\headrulewidth}{0pt} %Eemaldab päisest horisontaalse joone
\renewcommand{\footrulewidth}{0pt} %Eemaldab jalusest horisontaalse joone
\headheight = 57pt %Paneb paika päise laiuse (vastavalt kompilaatori soovitusele)
\footskip = 11pt %Jaluse ruum
\headsep = 0pt %Vähendab päise ja teksti vahelise kauguse nullini

\chead{ %Paigutab järgneva teksti päises keskele
 \textsc{\begin{Large} %Tekst suurtähtedega ja suuremaks
	tallinna tehnikaülikool\\
	\end{Large}}
	Infotehnoloogia teaduskond\\
	Arvutitehnika instituut
}
\vspace*{7 cm} %Tekitab lehe alguse ja teksti vahele tühja ala vastava laiusega

\begin{center} %Tekst keskele
ITC70LT\\[0cm]
Christopher David Raastad\\
\begin{LARGE}
\textsc{Euro 2.0 Digitaalne Valuuta Turvalisuse, Skaleeritamise, ja Privaatsuse Analüüs\\}
\end{LARGE}
Magister\\[2cm]
\end{center}

\begin{flushright} %Joondab teksti paremale
Alex Norta\\[0cm]  \todo[noline]{Tõlgi PhD eesti keelde}
PhD\\[0cm]
Associate Professor\\[0cm]  \todo[noline]{Tõlgi Associated Professor eesti keelde}
\end{flushright}

\cfoot{Tallinn 2015} %Lisab asukoha ja kuupäeva jalusesse
%\renewcommand{\headrulewidth}{0pt} %Eemaldab päisest horisontaalse joone
\pagebreak %Lehe lõpp


%----------------------------LIST OF TODOS----------------------------------
\listoftodos
\newpage

%---------------------------AUTHOR DECLARATION-------------------------
\selectlanguage{english}
\section*{\begin{center}
 Author’s declaration of originality
\end{center}}
I hereby certify that I am the sole author of this thesis. All the used materials, references to the literature and the work of others have been referred to. This thesis has not been presented for examination anywhere else.

Author: Christopher David Raastad

May 8th 2017
\pagebreak

%---------------------------ABSTRACT---------------------------------
\selectlanguage{estonian}
\section*{\begin{center}
Abstract
\end{center}}

\todo[inline]{Write English abstract \ldots}

If the thesis is written in English, the abstract is $\frac{1}{2}$ A4 long and the abstract in Estonian (\textit{Annotatsioon}) is of length 1 A4.

The last paragraph of abstract is obligatory and must be written accordingly:

\todo[inline]{Fill in English abstract thesis details \ldots}

The thesis is in English and contains [pages] pages of text, [chapters] chapters, [figures] figures, [tables] tables.

\pagebreak

%-----------------------------ANNOTATSIOON-----------------------------------
\section*{\begin{center}
Annotatsioon
\end{center}}

\todo[inline]{Kirjuta annotatsiooni eesti keeles \ldots}

Kui töö põhikeel on inglise keel, siis esitatakse annotatsioon (Abstract) inglise keeles mahuga $\frac{1}{2}$ A4 lehekülge ja annotatsioon eesti keeles mahuga vähemalt 1 A4 lehekülg.

Annotatsiooni viimane lõik on kohustuslik ja omab järgmist sõnastust:

\todo[inline]{Täitke eesti keele annotatsiooni lõputöö detailid \ldots}

Lõputöö on kirjutatud [mis keeles] keeles ning sisaldab teksti [lehekülgede arv] leheküljel, [peatükkide arv] peatükki, [jooniste arv] joonist, [tabelite arv] textit.

\pagebreak

%---------------------ABBREVIATIONS AND GLOSSARY OF TERMS---------------------
\selectlanguage{english}
\section*{\begin{center}
Table of abbreviations and terms
\end{center}}


\begin{tabular}{p{3 cm}ll} %Table where the first cell width is 3cm
BTC & Bitcoin currency code\\
AML & \textit{Anti Money Laundering}, processes implemented by Financial Institutions to hinder money laundering and comply with regulation\\
CTF & \textit{Counter Terrorist Financing}, processes implemented by Financial Institutions to hinder terrorist financing and comply with regulation\\
KYC & \textit{Know Your Customer}, proccesses of Financial institution to identify and verify customers' identities to aid AML and CTF\\
ISSRM & \textit{Information Systems Security Risk Management} \\
\end{tabular}

\todo[inline]{Continue adding to table of abbreviations and delete old ones\ldots}

\pagebreak

%----------------------------TABLE OF CONTENTS----------------------------------
\tableofcontents
\newpage

%----------------------LIST OF DRAWINGS-------------------------------
\listoffigures
\pagebreak

%----------------------LIST OF TABLES---------------------------------
\listoftables
\pagebreak

%-----------------------------CHAPTER 1 - INTRODUCTION-------------------------------
\section{Introduction} \label{sec:1}

Payments today are the laggard of the information age. While emails can be sent instantly for free anywhere in the world, money is either slow and/or expensive to move digitally. Bank transfers can take days, only work during business hours, and have high fees across borders. Card payments are instant, but subject merchants to high fees and chargeback risks. Paypal brought convenient payments to the web, but at a high cost to accept payments and move funds internationally. Fintech companies like Venmo and Square can create the illusion of fast payments, but still take days to settle in the background with the potential chargeback risks. This cost comes from the highly regulated centralised financial payments systems with little economic incentive to reduce fees. A usable digital currency would greatly alleviate this friction in transferring value that theoretically costs the economy an estimated 1\% of GDP annually \cite{kaarmann2013cost}.

Bitcoin was the first successful implementation of a decentralised digital currency. Nakamoto's peer to peer digital cash system completely sidestepped the existing financial system\cite{nakamoto2008bitcoin}. Its clever proof of work consensus protocol, economic incentives for nodes to maintain the network, pseudo anonymity, and irreversible transactions implemented on the public ledger Blockchain technology has been called ``the next technological revolution''\todo{citation, probably from Blockchain Revolution}. Hundreds of Altcoins followed Bitcoin's lead borrowing from the Blockchain technology to implement other flavours of cryptocurrency and digital asset management. Despite the rapid growth of the cryptocurrency ecosystem, everyday payments are a niche use case for digital currencies, which are more commonly used for investment and financial speculation\cite{Khairuddin:2016:EMB:2851581.2892500}. Merchants accepting Bitcoin payments are in fact immediately converting funds to a fiat currency for a non negligible fee. Bitcoin Exchanges are high cost gatekeepers between the cryptocurrency and financial systems.

What is missing today are incentives for parties on both sides of payments to hold assets end to end in digital currency. The cost and friction of moving between two financial systems, crypto and fiat, eliminates the perceived benefits of crypto currencies. The Euro 2.0 project aims to implement fiat currency on digital currency technology and reduce friction by eliminating unnecessary financial intermediaries. The unanimous adoption of Estonian ID card in Estonia and the Ethereum distributed application platform make a prototype system possible. This thesis introduces the Euro 2.0 system, its impact on stakeholders in the payments, and assesses the security, scalability, and privacy risks of the implementation to determine whether this system can in fact reasonably support a national digital currency.

\subsection{Existing Body of Knowledge} \label{ssec:1.1}
\todo[inline]{Existing body of knowledge to gap detection}


\subsection{Research Questions} \label{ssec:1.2}

This thesis explores the feasibility of the Euro 2.0 digital currency by answering the following research question:
\begin{quoting}
	\textbf{RQ: }\textit{How do we assure that the proposed Euro 2.0 digital currency does not have any major security, scalability, or privacy flaws compromising the usefulness of the monetary system?}
\end{quoting}

In which we explore in sub research questions:
\begin{quoting}
\textbf{RQ1: }\textit{How do we describe the relevant features needed for analysis of the Euro 2.0 digital currency?}
\end{quoting}
\begin{quoting}
\textbf{RQ2: }\textit{How does adoption of Euro 2.0 digital currency impact the relationship with money for key stakeholders: users, merchants, governments, and financial institutions?}
\end{quoting}
\begin{quoting}
\textbf{RQ3: }\textit{How do we assess the security, scalability, and privacy risks of Euro 2.0 digital currency system and potential impact on stakeholders?}
\end{quoting}

Answering \textbf{RQ} will result in one of two possible outcomes:
\begin{itemize}
	\item The Euro 2.0 system is free from significant risks in functioning as a digital monetary system.
	\item The Euro 2.0 system has one or more serious risks that need to be adressed before significant adoption.
\end{itemize}
In order to do a risk analysis, the key features of the Euro 2.0 system need to be identified and implementation details described by answering \textbf{RQ1}. Following a clear specification of the system, \textbf{RQ2} explores the current and future relationship with money of key stakeholders after adoption of Euro 2.0. This exploration of the stakeholders' relationship with money grounds the analysis of the impact of realised risks uncovered in \textbf{RQ3}. The severity and likelihood of realised risks to the monetary system lead to an answer to the original research question \textbf{RQ}.

\subsection{Research Methods} \label{ssec:1.3}
The thesis utilises methods from design science and Information Systems Security Risk Management (ISSRM) to approach the research questions. The main contributions are:
\begin{itemize}
	\item Analysis of stakeholders behaviour transition to Euro 2.0 system
	\item Risk analysis of the current Euro 2.0 system
	\item Recommendations to mitigate most severe risks
	\item Validation of risks and vulnerabilities on existing codebase test network
\end{itemize}

\subsection{Structure of the Thesis} \label{ssec:1.4}

\hypernameref{sec:2} will give an overview of the existing body of knowledge needed to understand the remaining chapters including: traditional finance, Bitcoin, Tether, Ethereum, Estonia ID, and research methods.
\hypernameref{sec:3} will describe the motivation, features, and implementations of the Euro 2.0 system and prototype needed for analysis. \hypernameref{sec:4} will analyse the impact of Euro 2.0 on stakeholders using design science research methods to better understand the socio-technical impact of the system. \hypernameref{sec:5} will perform the full risk analysis of the Euro 2.0 implementation from the angles of Security, Scalability, and Privacy using the ISSRM model and security design patterns research methods. \hypernameref{sec:6} will evaluate the findings of the risk analysis on the actual running prototype to verify the findings. And finally \hypernameref{sec:7} the conclusion will answer the research question and suggest future avenues of research.

\pagebreak

%--------------------CHAPTER 2 - BRIDGE-----------------
\section{Chapter 2: Bridge of Knowledge} \label{sec:2}

\todo[inline]{Write Chapter 2}

\subsection{Traditional Finance System}

\subsection{Bitcoin}

\subsection{Tether}

\subsection{CryptoCurrencies and Payments}
Existing crypto currencies not great for payments...

\subsection{Ethereum}
\begin{itemize}
	\item WTF is this thing?
	\item addresses
	\item messages
\end{itemize}

\subsection{Solidity Smart Contracts}
\begin{itemize}
	\item what can they do?
	\item data
	\item functions
	\item modifiers
	\item throw exceptions
\end{itemize}

\subsection{Estonia ID Card}

\subsection{Future Adoption Research}

\subsection{Security and Privacy Research}

\pagebreak

%--------------------CHAPTER 3-----------------
\section{Chapter 3} \label{sec:3}

\subsection{Introduction} \label{ssec:3.1}
This chapter introduces the Euro 2.0 digital currency system. The contents of this section is based of a prototype developed by the Euro 2.0 Foundation as a non profit initiative. All code is accessible in Github under the MIT License as indicated in \hyperref[ssec:a.1]{Appendix A.1}. The chapter answers the following research question:

\begin{quoting}
\textbf{RQ1:} \textit{How do we describe the relevant features needed for risk analysis of the Euro 2.0 digital currency?}
\end{quoting}

Which is broken down into the following sub-questions:
\begin{quoting}
	\textbf{RQ1.1: }\textit{What are the features of the Euro 2.0 digital currency?}
\end{quoting}{
\begin{quoting}
	\textbf{RQ1.2: }\textit{How are the decentralised features implemented?}
\end{quoting}{
\begin{quoting}
	\textbf{RQ1.3: }\textit{How are the centralised features implemented?}
\end{quoting}

Answering \textit{RQ1} lays the groundwork for further analysis in \hypernameref{sec:4} and \hypernameref{sec:5}. Section \ref{ssec:3.2} will briefly go through the motivation behind the Euro 2.0 digital currency system. Section \ref{ssec:3.3} will give an overview of the system architecture and features currently implemented answering \textit{RQ1.1}. Section \ref{ssec:3.4} will dive into the technical implementation of the decentralised features of the system on the Ethereum platform answering \textit{RQ1.2}. Finally Section \ref{ssec:3.5} will dive into the technical implementation of the centralised features built with traditional software development architecture answering \textit{RQ1.3}.

\subsection{Euro 2.0 Motivation} \label{ssec:3.2}

The idea of Euro 2.0 originated in 2014 from Kristo Käärmann's blogpost \textit{Government Backed Bitcoin}\cite{kaarmann2014government}. Following the success of Bitcoin, he saw an opportunity for a payments settled on a Government backed cryptocurrency, \textit{EuroCoin}, pegged one to one to real world Euro counterpart. Paypal cofounder Max Levchin supports the same idea, \textit{``I am confident that there will be a form of settlement that will be a crypto-currency''}\cite{pando2014levchin}. Bitcoin's short term fluctuations, functioning both as a commodity and a currency\todo{citation}, in value make it unreasonable for end to end holdings in a fiat based economy\todo{cite something about fluctuations}. Removing financial intermediaries in payments reduces the costs of transactions.

Even some governments have expressed interests in digital currency. The Swedish central bank debated whether to become the first to issue digital currency as a response to the country's move away from cash\cite{milne2016sweden}. The UK government pushed for a ``Call for Information'' on digital currencies in order to assess risks\cite{nermin2014ukcall} that was later responded by Citi bank's global Treasury and Trade Services (TTS) Technology and Innovation Team with a suggestion for government to create its own digital currency\cite{spaven2015ukcurrency}:

\begin{quoting}
	\textit{``The greatest benefits of digital currencies can be realised through the government issuing a digital form of legal tender. This currency would be less expensive, more efficient and provide greater transparency than current physical legal tender or electronic methods.''}
\end{quoting}

Following the rise in popularity of Ethereum, in mid 2016, another blogpost by Käärmann \textit{The future of money may be in the ether}\cite{kaarmann2016ether} described the rough idea of the system. I describe Euro 2.0 as solving the business problem:

\begin{quoting}
	\textit{How do I build a system to digitally store and transact fiat currency without a centralised for-profit financial intermediary conveniently, trustfully, and securely while satisfying government regulations?}
\end{quoting}

From this business problem derives the key pillars of Euro 2.0:

\begin{enumerate}
	\item Fiat Backed
	\item Decentralised Accounts and Transactions
	\item Identities of all Users
	\item Convenient Conversion of Fiat to and from Digital Euro
	\item Government can control monetary supply
	\item Features for Law Enforcement and AML
\end{enumerate}

\textit{Fiat Backed} (1) provides monetary stability to the system and builds trust with users and merchants to keep assets in the digital format. This eliminates the issue of constantly fluctuating daily value between real world and digital assets seen in crypto currencies today.

\textit{Decentralised Accounts and Transactions} (2) comes from the need of an impeccable record and process of transacting value that no centralised party can modify in a database. This is the key benefit and usecase of distributed ledger technology.

\textit{Identities of all Users} (3) builds trust in the system for users, merchants, and governments. If everyone is identified, any transaction can be legally ramified (assuming legislation arises supporting distributed ledger records in the court). The strong identity of Estonian ID card is a perfect fit for a digitally identifying mechanism and the main driver of the first prototype.

\textit{Convenient Conversion of Fiat to and from Digital Euro} (4) makes it easy for users to gradually start using the digital currency from existing financial infrastructure. At first this exchange gateway can be a private financial institution, similar to Tether\cite{tether2016whitepaper}, with a proof of reserve of the one to one Euro backing. Ideally this conversion service is hosted by the central bank itself to limit the number of financial intermediaries.

\textit{Government can control monetary supply} (5) follows from the ideal scenario of (4), the central bank can issue new money directly into the digital monetary system without a real world euro counterpart.

\textit{Features for Law Enforcement and AML} (6) is the most controversial aspect of the system. Bitcoin was made to sidestep the direct control of any third party, including government\cite{nakamoto2008bitcoin}. Euro 2.0 needs to be regulated from the start with full support of governments, which require means of necessary intervention that typically happen today via cooperation with traditional financial institutions\todo{cite relationship between banks and government}.

\subsection{Euro 2.0 Features} \label{ssec:3.3}

The following section describes and categorises the main features of the Euro 2.0 system. Features are enumerated with \textbf{XY\#} where:
\begin{itemize}
	\item \textbf{X} is an abbreviation of the domain of the feature, described in the \hypernameref{sssec:3.3:domains} section
	\item \textbf{Y} is client or server, where client is a user of the system, and server is some centralised or decentralised server infrastructure
	\item \textbf{\#} is the number of a feature for a given \textbf{XY}
\end{itemize}

\subsubsection{Domains} \label{sssec:3.3:domains}

\hypertableref{tab:domains} gives an overview of the domains of Euro 2.0 used to group features described in later sections.

\begin{center}
\begin{tabular}{ | l | c | p{10cm} | }
 \hline
 Domain & Abbr. & Description \\
 \hline
 \textbf{Account} & A & mechanism of storage of value in the system
 \\ \hline
 \textbf{Transaction} & T & mechanism of transfer of value in the system
 \\ \hline
 \textbf{Network} & N & decentralised infrastructure settling EUR2 payments
 \\ \hline
 \textbf{Identity} & I & features linking real life human identity to users
 \\ \hline
 \textbf{Reserve} & R & features regarding the exchange of value between traditional Euro (EUR) in the financial system and digital Euro (EUR2)
 \\ \hline
 \textbf{Enforcement} & E & features for government regulators and law enforcement to police the system for misuse and AML
 \\ \hline
 \textbf{Convenience} & C & features for improving user experience and expanding use cases
 \\ \hline
\end{tabular}
\end{center}
\captionof{table}{Domains of Euro 2.0}
\label{tab:domains}

\hypertableref{tab:centVsDecent} gives an overview of the distribution of centralised and decentralised domains of the Euro 2.0 system.

\begin{center}
\begin{tabular}{ |c|c| }
 \hline
 Centralised & Decentralised \\
 \hline
 Identity & Account \\
 Reserve & Transaction \\
 Enforcement & Network \\
 Convenience & \\
 \hline
\end{tabular}
\end{center}
\captionof{table}{Centralised vs. Decentralised Domains of Euro 2.0}
\label{tab:centVsDecent}

\subsubsection{Entities} \label{sssec:3.3:entities}

\hypertableref{tab:entities} clarifies key data entities used throughout the description of features.

\begin{center}
\begin{tabular}{ | p{3cm} | p{12cm} | }
 \hline
 Entity & Description
 \\ \hline\hline
 Address & Ethereum externally controlled address (i.e. public key)
 \\ \hline
 Key & Ethereum address private key needed to use funds in address
 \\ \hline
 Account & All verified Ethereum addresses for an ID
 \\ \hline
 Account Balance & Sum of total balance of an account
 \\ \hline
 ID & Estonian ID code of an individual (citizen, resident, e-resident)
 \\ \hline
 ID Name & Name of person associated with an Estonian ID Code
 \\ \hline
 EUR & Euro currency in an Estonian bank account
 \\ \hline
 EUR2 & Euro currency in the Euro 2.0 digital form, balance of an address in the Euro 2.0 Ethereum contract
 \\ \hline
 Wallet & Client program (app or mobile web) used as an interface to Euro 2.0 system, accessible keys to addresses
 \\ \hline
 Wallet Password & Password used to encrypt and decrypt local wallet
 \\ \hline
 Backup Password & Password used to encrypt and decrypt keys and address to send to backup server
 \\ \hline
\end{tabular}
\end{center}
\captionof{table}{Euro 2.0 Entities}
\label{tab:entities}

\subsubsection{Processes} \label{sssec:3.3:processes}

\hypertableref{tab:processes} clarifies different processes used throughout the description of features.

\begin{center}
\begin{tabular}{ | p{3cm} | p{12cm} | }
 \hline
 Process & Description
 \\ \hline\hline
 Approval & Link Ethereum address with the an Estonian ID number via ID card certificate verification
 \\ \hline
 Freeze & Stop all transactions from an ID's address
 \\ \hline
 Seize & Remove all asets from an ID's account and send them to the court's account
 \\ \hline
 Name checks & AML PEP and Sanction list check on real name of person
 \\ \hline
\end{tabular}
\end{center}
\captionof{table}{Euro 2.0 Processes}
\label{tab:processes}

\subsubsection{Account Features} \label{sssec:3.3:accounts}

Client
\begin{itemize}
	\item \textbf{AC1} - Create address
	\item \textbf{AC2} - Approve address
	\item \textbf{AC3} - View account balance
\end{itemize}

Server
\begin{itemize}
	\item \textbf{AS1} - Check external source for valid ID
	\item \textbf{AS2} - Approve address
	\item \textbf{AS3} - Map ID to address
	\item \textbf{AS4} - Aggregate account balance
\end{itemize}

\subsubsection{Transaction Features} \label{sssec:3.3:transactions}

Client
\begin{itemize}
	\item \textbf{TC1} - Send EUR2 to an address
	\item \textbf{TC2} - Send EUR2 to an ID
	\item \textbf{TC3} - Receive EUR2 to an address
	\item \textbf{TC4} - Receive EUR2 to an ID
\end{itemize}

Server
\begin{itemize}
	\item \textbf{TS1} - Resolve ID to an address
	\item \textbf{TS2} - Escrow EUR2 to an ID
	\item \textbf{TS3} - Release escrow EUR2 to an address for an ID
	\item \textbf{TS4} - Verify transaction
\end{itemize}

\subsubsection{Reserve Features} \label{sssec:3.3:reserve}

Client
\begin{itemize}
	\item \textbf{RC1} - Send EUR to Reserve bank account
	\item \textbf{RC2} - Receive EUR2 from Reserve to an address
	\item \textbf{RC3} - Send EUR2 to a EUR bank account via the Reserve
\end{itemize}

Server
\begin{itemize}
	\item \textbf{RS1} - Resolve address from ID of received EUR transaction
	\item \textbf{RS2} - Send EUR2 to an address
	\item \textbf{RS3} - Create EUR2
	\item \textbf{RS4} - Receive EUR2 from an address
	\item \textbf{RS5} - Send EUR to a bank account
	\item \textbf{RS6} - Destroy EUR2
\end{itemize}

\subsubsection{Enforcement Features} \label{sssec:3.3:enforcement}

Server
\begin{itemize}
	\item \textbf{ES1} - Freeze account
	\item \textbf{ES2} - Seize account funds to an address
	\item \textbf{ES3} - Name checks on ID names
	\item \textbf{ES4} - Map address to ID
\end{itemize}

\subsubsection{Convenience Features} \label{sssec:3.3:backup}

\paragraph{Wallet Security}
Client
\begin{itemize}
	\item \textbf{CC1} - Set up wallet encryption
	\item \textbf{CC2} - Decrypt wallet
\end{itemize}

\paragraph{Key Backup}

Client
\begin{itemize}
	\item \textbf{CC3} - Encrypt addresses and keys
	\item \textbf{CC4} - Sync addresses and keys to backup server
	\item \textbf{CC5} - Restore addresses and keys from backup server
	\item \textbf{CC6} - Decrypt addresses and keys
\end{itemize}

Server
\begin{itemize}
	\item \textbf{CS1} - Check ID is approved
	\item \textbf{CS2} - Save encrypted addresses and keys
	\item \textbf{CS3} - Load encrypted addresses and keys
\end{itemize}

\paragraph{Transfer Info}

Client
\begin{itemize}
	\item \textbf{CC7} - encrypt transfer details
	\item \textbf{CC8} - Save transfer details for a transaction
	\item \textbf{CC9} - Load transfer details for a transaction
\end{itemize}

Server
\begin{itemize}
	\item \textbf{CS4} - Save transfer details for a transaction
	\item \textbf{CS5} - Load transfer details for a transaction
\end{itemize}


\subsection{Euro 2.0 Decentralised Implementation} \label{ssec:3.4}

This section walks through the decentralised implementation of the Euro 2.0 system on the Ethereum distributed applications platform. The main container of code deployment is the \textit{Contract}. Once deployed, the contract is on the Blockchain forever processing messages (transactions) sent to it unless an optional destroy function is implemented and called. All function calls and data modifications are irreversible and publicly visible while being executed on the 1000s of computers on the Ethereum network maintaining the Ethereum Blockchain.

The system is comprised of of the main administrator contract (\textit{CryptoFiat}), shared shared contracts (\textit{Constants}, \textit{Relay}, \textit{Data}, \textit{InternalData}), and the upgradable subcontracts providing system functionality (\textit{Accounts}, \textit{Approving}, \textit{Reserve}, \textit{Enforcement}, \textit{AccountRecovery}, \textit{Delegation}). These decentralised contracts function as gateways to the shared data of the Euro 2.0 system used together with centralised servers and clients explained in \hyperref[ssec:3.5]{Section \ref{ssec:3.5}}.

Some simplifications are made describing the system. Little attention is payed to events as its not relevan for describing the providers of functionality. Data types are simplified (i.e. uint256 is just written int). Complete specification can be viewed in the code.

\todo[inline]{Diagram with inheritance structure of all contracts}

\subsubsection{CryptoFiat Contract} \label{sssec:3.4:cryptofiat}

The heart of decentralised Euro 2.0 is the \textit{CryptoFiat} contract. This contract functions as an administrator, managing the master address and the references to all deployed contracts comprising the main functionality of the Euro 2.0 system.

\begin{center}
\resizebox{\textwidth}{!}{
\begin{tabular}{ | l | l | p{9cm} | }
 \hline
 Name & Type & Description
 \\ \hline\hline
 \code{masterAccount} & \code{address} & Master ethereum account whose allowed to do operations on the CryptoFiat contract. Set in the constructor and changeable by master account owner.
 \\ \hline
 \code{contractAddress} & \code{int => address} & Stores mapping from subcontract id to deployed address of active subcontract.
 \\ \hline
 \code{contractId} & \code{address => int} & Stores mapping from active subcontract address to id.
 \\ \hline
 \code{contracts} & \code{address[]} & Array with the address of all contracts ever added to the Euro 2.0 system. It's contents is never cleared.
 \\ \hline
\end{tabular}}
\end{center}
\captionof{table}{CryptoFiat contract data}
\label{tab:cryptoFiatData}

\begin{center}
\resizebox{\textwidth}{!}{
\begin{tabular}{ | l | l | p{9cm} | }
 \hline
 Function & Args & Description
 \\ \hline\hline
 \code{contractActive} & \code{address addr} & Returns \code{bool} whether the subcontract at address \code{addr} is active.
 \\ \hline
 \code{contractsLength} & & Returns the length of \code{contracts}, i.e. the number of all contracts ever deployed.
 \\ \hline
 \code{appointMasterAccount} & \code{address next} & Sets \code{masterAccount} to \code{next} and hence gives up control of the CryptoFiat contract. Can only be called by master account owner.
 \\ \hline
 \code{upgrade} & \specialcell{\code{int id},\\ \code{address next}} & Upgrades the active contract with \code{id} to the address \code{next}. Only owner of master account can call this. \code{prev} cannot be the same contract as \code{next}. \code{next} cannot be an already active contract. \code{contractAddress}, \code{contractId}, and \code{contracts} are updated appropriately.
 \\ \hline
\end{tabular}}
\end{center}
\captionof{table}{CryptoFiat contract functions}
\label{tab:cryptoFiatFunctions}

\subsubsection{Shared Contracts: Constants, Relay, Data, InternalData} \label{sssec:3.4:shared}

The code in the shared contracts in Euro 2.0 are exposed ultimately in the \textit{InternalData} contract interface that heavily calls \textit{Data contract}. \textit{InternalData} contract acts as an abstract Contract inherited by all the sub contracts providing functionality. Only the \textit{Data} contract is actually constructed and deployed on the Blockchain, the other shared contracts are simply organising code. \textit{InternalData} inherits from \textit{Constants} and \textit{Relay}. While \textit{Relay} inherits from \textit{Constants} and references \textit{Data} for use by \textit{InternalData}. \textit{Data} inherits from \textit{Relay}.

\paragraph{Constants Contract}
The \textit{Constants} contract holds values for contract ids (used in contract deployment mapping), bucket identifiers (used in data storage keys), account states (used in accounts and transactions), and events. There is no functionality, just convenient constants to organize data and contract access used in other contracts.

\begin{center}
\begin{tabular}{ | l | l | p{10cm} | }
 \hline
 Contract Name & Id & Description
 \\ \hline\hline
 DATA & 1 & Contains data storage for all Euro 2.0 contracts
 \\ \hline
 ACCOUNTS & 2 & Contract regarding account operations
 \\ \hline
 APPROVING & 3 & Contract approving accounts based on verified ID
 \\ \hline
 RESERVE & 4 & Contract able to increase, decrease, and transfer supply
 \\ \hline
 ENFORCEMENT & 5 & Contract with law enforcement operations
 \\ \hline
 ACCOUNT\_RECOVERY & 6 & Contract assigning account recovery options
 \\ \hline
 DELEGATION & 7 & ???
 \\ \hline
\end{tabular}
\end{center}
\captionof{table}{Constant contract subcontract ids}
\label{tab:constantSubcontractIds}

\begin{center}
\begin{tabular}{ | l | l | p{10cm} | }
 \hline
 Contract Name & Id & Description
 \\ \hline\hline
 STATUS & 1 & Account states
 \\ \hline
 BALANCE & 2 & Account balance
 \\ \hline
 DELEGATED\_TRANSFER\_NONCE & 3 & ???
 \\ \hline
 RECOVERY\_ACCOUNT & 4 & Recovery account assignments
 \\ \hline
 TOTAL\_SUPPLY & 5 & Total supply issued by reserve
 \\ \hline
\end{tabular}
\end{center}
\captionof{table}{Constant contract bucket identifiers}
\label{tab:constantBucketIds}

\begin{center}
\begin{tabular}{ | l | l | p{10cm} | }
 \hline
 State & Id & Description
 \\ \hline\hline
 APPROVED & 1 & Account has a linked ID approved and can send funds.
 \\ \hline
 CLOSED & 2 & Account is closed (by owner or law enforcement) and cannot receive funds.
 \\ \hline
 FROZEN & 4 & Account is frozen by law enforcement and cannot send funds.
 \\ \hline
\end{tabular}
\end{center}
\captionof{table}{Constant contract account state}
\label{tab:constantAccountStates}

\paragraph{Relay Contract}

The \textit{Relay} contract holds convenience methods for subcontracts to access each other. The functions in \hypertableref{tab:relayFunctions} use the reference to \textit{CryptoFiat} in \hypertableref{tab:relayData} to provide a reference to all deployed subcontracts. There is also convenient restriction modifiers listed in \hypertableref{tab:relayModifiers}.

\begin{center}
\resizebox{\textwidth}{!}{
\begin{tabular}{ | l | l | p{9cm} | }
 \hline
 Name & Type & Description
 \\ \hline\hline
 \code{cryptoFiat} & \code{address} & Reference to main \textit{CryptoFiat} contract deployment used to resolve address to all other contracts
 \\ \hline
\end{tabular}}
\end{center}
\captionof{table}{Relay contract data}
\label{tab:relayData}

\begin{center}
\resizebox{\textwidth}{!}{
\begin{tabular}{ | l | l | p{9cm} | }
 \hline
 Modifier &  Description
 \\ \hline\hline
 \code{onlyMasterAccount} & Restricts function to only be usable by master account by checking \code{msg.sender}.
 \\ \hline
 \code{onlyContracts} & Restricts function to only be usable by active contracts.
 \\ \hline
\end{tabular}}
\end{center}
\captionof{table}{Relay contract modifiers}
\label{tab:relayModifiers}

\begin{center}
\resizebox{\textwidth}{!}{
\begin{tabular}{ | l | l | p{9cm} | }
 \hline
 Function & Args & Description
 \\ \hline\hline
 \code{switchCryptoFiat} & \code{address next} & Sets the address of \textit{CryptoFiat} contract. Restricted to \code{onlyMasterAccount}.
 \\ \hline
 \code{contractAddress} & \code{int id} & Returns the address of subcontract of \code{id} listed in \hypertableref{tab:constantSubcontractIds}.
 \\ \hline
 \code{accounts} & & Returns reference to \code{ACCOUNTS} contract.
 \\ \hline
 \code{data} & & Returns reference to \code{DATA} contract.
 \\ \hline
 \code{approving} & & Returns reference to \code{APPROVING} contract.
 \\ \hline
 \code{reserve} & & Returns reference to \code{RESERVE} contract.
 \\ \hline
 \code{accountRecovery} & & Returns reference to \code{ACCOUNT\_RECOVERY} contract.
 \\ \hline
 \code{delegation} & & Returns reference to \code{DELEGATION} contract.
 \\ \hline
\end{tabular}}
\end{center}
\captionof{table}{Relay contract functions}
\label{tab:relayFunctions}

\paragraph{Data Contract}

The \textit{Data} contract provides the main data access layer for all subcontracts which is accessed by convenience methods in the \textit{InternalData} contract. No subcontracts work with this contract directly. The main data object is defined in \hypertableref{tab:dataData} and functions on this data in \hypertableref{tab:dataFunctions}.

\begin{center}
\resizebox{\textwidth}{!}{
\begin{tabular}{ | l | l | p{9cm} | }
 \hline
 Name & Type & Description
 \\ \hline\hline
 \code{\_data} & \code{map: bytes32 => bytes32} & The shared data storage structure for all subcontracts
 \\ \hline
\end{tabular}}
\end{center}
\captionof{table}{Data contract data}
\label{tab:dataData}

\begin{center}
\resizebox{\textwidth}{!}{
\begin{tabular}{ | l | l | p{12cm} | }
 \hline
 Function & Args & Description
 \\ \hline\hline
 \code{set} & \specialcell{\code{int bucket},\\ \code{bytes32 key},\\ \code{bytes32 value}} & Saves the data \code{value} with the hash key \code{sha3(bucket, key)}, with \code{bucket} referring to the bucket id list in \hypertableref{tab:constantBucketIds} and key being an arbitrary value. Restricted to \code{onlyContracts} callers.
 \\ \hline
 \code{get} & \code{int id} & Returns \code{byte32} value stored in hash key \code{sha3(bucket, key)}, with bucket id being from the list in \hypertableref{tab:constantBucketIds}.
 \\ \hline
\end{tabular}}
\end{center}
\captionof{table}{Data contract functions}
\label{tab:dataFunctions}

\paragraph{InternalData Contract}

\textit{InternalData} is the main contract for accessing data storage used by all subcontracts. All functions here are marked \code{internal}, meaning they can only be accessed from within a contract. All subcontracts extend \textit{InternalData}. Data access functions are listed in \hypertableref{tab:internalDataAccessFunctions}, account status functions are listed in \hypertableref{tab:internalDataAccountStatusFunctions}, modifiers in \hypertableref{tab:internalDataModifiers}, and account functions in \hypertableref{tab:internalDataAccountFunctions}.

\begin{center}
\resizebox{\textwidth}{!}{
\begin{tabular}{ | l | l | p{12cm} | }
 \hline
 Function & Args & Description
 \\ \hline\hline
 \code{\_balanceOf} & \code{address addr} & Returns \code{int} BALANCE stored for address \code{addr}.
 \\ \hline
 \code{\_setBalanceOf} & \specialcell{\code{address addr},\\ \code{int value}} & Stores BALANCE of \code{value} for \code{addr}.
 \\ \hline
 \code{\_statusOf} & \code{address addr} & Returns \code{int} account STATUS of address \code{addr} as listed in status codes of \hypertableref{tab:constantAccountStates}.
 \\ \hline
 \code{\_setStatusOf} & \specialcell{\code{address addr},\\ \code{int value}} & Stores account STATUS of \code{value} for \code{addr}.
 \\ \hline
 \code{\_delegatedTransferNonceOf} & \code{address addr} & Returns \code{int} of last nonce used in delegatedTransfer for address \code{addr}.
 \\ \hline
 \code{\_setDelegatedTransferNonceOf} & \specialcell{\code{address addr},\\ \code{int value}} & Stores last nonce used \code{value} in delegatedTransfer for address \code{addr}.
 \\ \hline
 \code{\_recoveryAccountOf} & \code{address addr} & Returns the \code{address} recovery account set for address \code{addr}.
 \\ \hline
 \code{\_setRecoveryAccountOf} & \specialcell{\code{address addr},\\ \code{address value}} & Stores recovery account \code{value} for address \code{addr}.
 \\ \hline
 \code{\_totalSupply} & & Returns total supply \code{int} of tokens in circulation.
 \\ \hline
 \code{\_setTotalSupply} & \code{int value} & Stores \code{value} of total tokens in circulation.
 \\ \hline
\end{tabular}}
\end{center}
\captionof{table}{InternalData contract account status functions}
\label{tab:internalDataAccessFunctions}

\begin{center}
\resizebox{\textwidth}{!}{
\begin{tabular}{ | l | l | p{12cm} | }
 \hline
 Function & Args & Description
 \\ \hline\hline
 \code{\_isApproved} & \code{address account} & Returns \code{bool} whether \code{account} is APPROVED.
 \\ \hline
 \code{\_isClosed} & \code{address account} & Returns \code{bool} whether \code{account} is CLOSED.
 \\ \hline
 \code{\_isFrozen} & \code{address account} & Returns \code{bool} whether \code{account} is FROZEN.
 \\ \hline
\end{tabular}}
\end{center}
\captionof{table}{InternalData account status functions}
\label{tab:internalDataAccountStatusFunctions}

\begin{center}
\resizebox{\textwidth}{!}{
\begin{tabular}{ | l | l | p{12cm} | }
 \hline
 Function & Args & Description
 \\ \hline\hline
 \code{canSend} & \code{address account} & Validates if \code{account} can send by throwing an exception if it is not approved, frozen, or the address 0.
 \\ \hline
 \code{assertSend} & \code{address account} & Function form of \code{canSend}.
 \\ \hline
 \code{canReceive} & \code{address account} & Validates if \code{account} can receive by throwing an exception if it is closed or the address 0.
 \\ \hline
 \code{assertReceive} & \code{address account} & Function form of \code{canReceive}.
 \\ \hline
\end{tabular}}
\end{center}
\captionof{table}{InternalData modifiers}
\label{tab:internalDataModifiers}

\begin{center}
\resizebox{\textwidth}{!}{
\begin{tabular}{ | l | l | p{12cm} | }
 \hline
 Function & Args & Description
 \\ \hline\hline
 \code{\_withdraw} & \specialcell{\code{address account},\\ \code{int amount}} & Withdraws \code{amount} from balance of \code{account}. Throws exception if \code{amount} is more than balance.
 \\ \hline
 \code{\_deposit} & \specialcell{\code{address account},\\ \code{int amount}} & Deposits \code{amount} into balance for \code{account}. Throws exception if amount to withdraw plus balance is less than balance (overflow).
 \\ \hline
\end{tabular}}
\end{center}
\captionof{table}{InternalData account functions}
\label{tab:internalDataAccountFunctions}

\subsubsection{Accounts Contract} \label{sssec:3.5:accounts}
The \textit{Accounts} contract exposes all the functions, listed in \hypertableref{tab:accountsFunctions}, necessary from \textit{InternalData} to work with accounts and balances.

\begin{center}
\resizebox{\textwidth}{!}{
\begin{tabular}{ | l | l | p{12cm} | }
 \hline
 Function & Args & Description
 \\ \hline\hline
 \code{balanceOf} & \code{address account} & Returns \code{int} balance of \code{account}.
 \\ \hline
 \code{statusOf} & \code{address account} & Returns \code{int} status of \code{account} as defined in table \hypertableref{tab:constantAccountStates}.
 \\ \hline
 \code{isApproved} & \code{address account} & Returns \code{bool} whether \code{account} is approved (can send funds).
 \\ \hline
 \code{isClosed} & \code{address account} & Returns \code{bool} whether \code{account} is closed (can not receive funds).
 \\ \hline
 \code{isFrozen} & \code{address account} &  Returns \code{bool} whether \code{account} is frozen (can not send funds).
 \\ \hline
 \code{transfer} & \specialcell{\code{address destination},\\ \code{int amount}} & Withdraws \code{amount} from caller \code{msg.sender} and deposits \code{amount} into \code{destination}. Restricted to \code{msg.sender} that \code{canSend} and \code{destination} that \code{canReceive}.
 \\ \hline
\end{tabular}}
\end{center}
\captionof{table}{Accounts functions}
\label{tab:accountsFunctions}

\subsubsection{Approving Contract} \label{sssec:3.5:approving}
\textit{Approving} contract exposes the functions necessary for approving accounts to transact in Euro 2.0. Only the \code{accountApprover}, \hypertableref{tab:approvingData}, can approve accounts. Modifiers, \hypertableref{tab:approvingModifiers} restrict access to functions, \hypertableref{tab:approvingFunctions}, to \code{accountApprover}.

\begin{center}
\resizebox{\textwidth}{!}{
\begin{tabular}{ | l | l | p{9cm} | }
 \hline
 Name & Type & Description
 \\ \hline\hline
 \code{accountApprover} & \code{address} & Account approver address.
 \\ \hline
\end{tabular}}
\end{center}
\captionof{table}{Approving contract data}
\label{tab:approvingData}

\begin{center}
\resizebox{\textwidth}{!}{
\begin{tabular}{ | l | l | p{12cm} | }
 \hline
 Function & Args & Description
 \\ \hline\hline
 \code{onlyAccountApprover} & & Validates if \code{msg.sender} is \code{accountApprover} otherwise throws an exception.
 \\ \hline
\end{tabular}}
\end{center}
\captionof{table}{Approving contract modifiers}
\label{tab:approvingModifiers}

\begin{center}
\resizebox{\textwidth}{!}{
\begin{tabular}{ | l | l | p{12cm} | }
 \hline
 Function & Args & Description
 \\ \hline\hline
 \code{appointAccountApprover} & \code{address next} & Assigns address of \code{accountApprover} to \code{next}, removing approving privileges to the old address. Restricted to \code{onlyAccountApprover}.
 \\ \hline
 \code{approveAccount} & \code{address account} & Sets status of \code{account} to APPROVED to be able to send money. Restricted to \code{onlyAccountApprover}.
 \\ \hline
 \code{approveAccounts} & \code{address[] accounts} & Sets status of each account in \code{accounts} to APPROVED to be able to send money. Restricted to \code{onlyAccountApprover}.
 \\ \hline 
 \code{closeAccount} & \code{address account} & Sets status of \code{account} to CLOSED to prevent the account from receiving money. Restricted to \code{onlyAccountApprover}.
 \\ \hline
\end{tabular}}
\end{center}
\captionof{table}{Approving contract functions}
\label{tab:approvingFunctions}

\subsubsection{Reserve Contract} \label{sssec:3.5:reserve}

The \textit{Reserve} contract exposes the functions to increase and decrease supply of EUR2 in the Euro 2.0. Only the \code{reserveBank} account, \hypertableref{tab:reserveData}, can perform these tasks. The modifier in \hypertableref{tab:reserveModifiers} restrict access to functions listed in \hypertableref{tab:reserveFunctions} to \code{reserveBank}.

\begin{center}
\resizebox{\textwidth}{!}{
\begin{tabular}{ | l | l | p{9cm} | }
 \hline
 Name & Type & Description
 \\ \hline\hline
 \code{reserveBank} & \code{address} & Reserve bank address.
 \\ \hline
\end{tabular}}
\end{center}
\captionof{table}{Reserve contract data}
\label{tab:reserveData}

\begin{center}
\resizebox{\textwidth}{!}{
\begin{tabular}{ | l | l | p{12cm} | }
 \hline
 Function & Args & Description
 \\ \hline\hline
 \code{onlyReserveBank} & & Validates if \code{msg.sender} is \code{reserveBank} otherwise throws an exception.
 \\ \hline
\end{tabular}}
\end{center}
\captionof{table}{Reserve contract modifiers}
\label{tab:reserveModifiers}

\begin{center}
\resizebox{\textwidth}{!}{
\begin{tabular}{ | l | l | p{12cm} | }
 \hline
 Function & Args & Description
 \\ \hline\hline
 \code{appointReserveBank} & \code{address next} & Assigns address of \code{reserveBank} to \code{next}, removing  privileges of the old address. Restricted to \code{onlyReserveBank}.
 \\ \hline
 \code{totalSupply} & & Returns \code{int} total supply of Euro 2.0 system. No restrictions on calling this method. 
 \\ \hline
 \code{increaseSupply} & \code{int amount} & Increases supply by \code{amount} and deposits \code{amount} of newly created EUR2 to the \code{reserveBank}. Restricted to \code{onlyReserveBank} and \code{reserveBank} \code{canReceive}. Throws exception if supply plus amount overflows.
 \\ \hline 
 \code{decreaseSupply} & \code{int amount} & Decreases supply by \code{amount} and withdraws \code{amount} from \code{reserveBank}, destroying the EUR2. Restricted to \code{onlyReserveBank} and \code{reserveBank} \code{canSend}. Throws exception if supply is less than amount.
 \\ \hline
\end{tabular}}
\end{center}
\captionof{table}{Reserve contract functions}
\label{tab:reserveFunctions}

\subsubsection{Enforcement Contract} \label{sssec:3.5:enforcement}

The \textit{Enforcement} contract exposes the functions for law enforcement to freeze and seize funds. The two roles are the law enforcer, who can do the actions to freeze and seize funds, and account designator, who can control the account where the funds can be seized, described in \hypertableref{tab:enforcementData}. Modifiers in \hypertableref{tab:enforcementModifiers} restrict access to functions based on roll. The functions are listed in \hypertableref{tab:enforcementFunctions}.

\begin{center}
\resizebox{\textwidth}{!}{
\begin{tabular}{ | l | l | p{9cm} | }
 \hline
 Name & Type & Description
 \\ \hline\hline
 \code{lawEnforcer} & \code{address} & Law enforcer address.
 \\ \hline
 \code{accountDesignator} & \code{address} & Account designator address.
 \\ \hline
 \code{account} & \code{address} & Law enforcement account.
 \\ \hline
\end{tabular}}
\end{center}
\captionof{table}{Enforcement contract data}
\label{tab:enforcementData}

\begin{center}
\resizebox{\textwidth}{!}{
\begin{tabular}{ | l | l | p{12cm} | }
 \hline
 Function & Args & Description
 \\ \hline\hline
 \code{onlyLawEnforcer} & & Validates if \code{msg.sender} is \code{lawEnforcer} otherwise throws an exception.
 \\ \hline
 \code{onlyAccountDesignator} & & Validates if \code{msg.sender} is \code{accountDesignator} otherwise throws an exception.
 \\ \hline
\end{tabular}}
\end{center}
\captionof{table}{Enforcement contract modifiers}
\label{tab:enforcementModifiers}

\begin{center}
\resizebox{\textwidth}{!}{
\begin{tabular}{ | l | l | p{12cm} | }
 \hline
 Function & Args & Description
 \\ \hline\hline
 \code{appointLawEnforcer} & \code{address next} & Assigns address of \code{lawEnforcer} to \code{next}, removing privileges of the old address. Restricted to \code{onlyLawEnforcer}.
 \\ \hline
 \code{appointAccountDesignator} & \code{address next} & Assigns address of \code{accountDesignator} to \code{next}, removing privileges of the old address. Restricted to \code{onlyAccountDesignator}.
 \\ \hline
 \code{withdraw} & \specialcell{\code{address from}, \\ \code{int amount}} & Withdraws \code{amount} from account \code{from} and deposits \code{amount} into law enforcement \code{account}. Restricted to \code{onlyLawEnforcer} and \code{account} \code{canReceive}.
 \\ \hline
 \code{freezeAccount} & \code{address target} & Sets status of \code{target} account to FROZEN so it can no longer send funds. Restricted to \code{onlyLawEnforcer}.
 \\ \hline
 \code{unFreezeAccount} & \code{address target} & Removes FROZEN status of \code{target} account so it can send funds again. Restricted to \code{onlyLawEnforcer}.
 \\ \hline
 \code{designateAccount} & \code{address account} & Sets law enforcement \code{account} to given \code{account}. Restricted to \code{onlyAccountDesignator} and \code{account} \code{canReceive}.
 \\ \hline
\end{tabular}}
\end{center}
\captionof{table}{Enforcement contract functions}
\label{tab:enforcementFunctions}

\subsubsection{AccountRecovery Contract} \label{sssec:3.5:accountRecovery}

The \textit{AccountRecovery} exposes functionality intended for an account recovery mechanism. An account owner can designate a trusted party to withdraw all funds and close the account. The two functions enabling this functionality are listed in \hypertableref{tab:accountRecoveryFunctions}.

\begin{center}
\resizebox{\textwidth}{!}{
\begin{tabular}{ | l | l | p{12cm} | }
 \hline
 Function & Args & Description
 \\ \hline\hline
 \code{designateRecoveryAccount} & \code{address recoveryAccount} & Sets the recovery account for \code{msg.sender} to address \code{recoveryAccount}, replacing an existing \code{recoveryAccount}. To remove a recovery account the address 0 can be set.
 \\ \hline
 \code{recoverBalance} & \specialcell{\code{address from}, \\ \code{address into}} & Withdraws all funds from account \code{from} and deposits all funds to account \code{into} and then closing account \code{from} by setting its status to CLOSED. Restricted to account \code{from} \code{canSend} and account \code{into} \code{canReceive}.
 \\ \hline
\end{tabular}}
\end{center}
\captionof{table}{AccountRecovery contract functions}
\label{tab:accountRecoveryFunctions}

\subsubsection{Delegation Contract} \label{sssec:3.5:delegation}

\todo[inline]{The \textit{Delegation} contract is magic...}
Its functions are listed in \hypertableref{tab:delegationFunctions}}.

\begin{center}
\resizebox{\textwidth}{!}{
\begin{tabular}{ | l | l | p{12cm} | }
 \hline
 Function & Args & Description
 \\ \hline\hline
 \code{nonceOf} & \code{address account} & Sets delegated transfer nonce of \code{account}.
 \\ \hline
 \code{recoverSigner} & \specialcell{\code{bytes32 hash}, \\ \code{bytes signature}} & Magic...
 \\ \hline
 \code{transfer} 
    & \specialcell{
 		% transfer request
        \code{uint256 nonce}, \\
		\code{address destination}, \\
		\code{uint256 amount}, \\
		\code{uint256 fee}, \\
        % transfer request signed by source
        \code{bytes signature}, \\
        % whom to pay for fulfilling transfer
        \code{address delegate}
 	   } 
    & Magic...
 \\ \hline
 \code{recoverXfer} & \specialcell{\code{bytes data}, \\ \code{uint offset}} & Magic ...
 \\ \hline
 \code{multitransfer} 
 	& \specialcell{\code{uint256 count}, \\ \code{bytes transfers}, \\ \code{address delegate}}
	& Magic ...
 \\ \hline
\end{tabular}}
\end{center}
\captionof{table}{Delegation contract functions}
\label{tab:delegationFunctions}

\subsection{Euro 2.0 Centralised Implementation} \label{ssec:3.5}

\pagebreak

%--------------------CHAPTER 4-----------------
\section{Chapter 4} \label{sec:4}

\todo[inline]{Write Chapter 4}

\begin{quoting}
	\textbf{RQ2:} How does adoption of Euro 2.0 digital currency impact the relationship with money for key stakeholders: users, merchants, governments, and financial institutions?
\end{quoting}

\begin{quoting}
	\textbf{RQ2.1: }\textit{What is the current relationship of money for stakeholders?}
\end{quoting}
\begin{quoting}
	\textbf{RQ2.2: }\textit{What changes in the relationship of money for stakeholders after adopting the Euro 2.0 digital currency?}
\end{quoting}
\begin{quoting}
	\textbf{RQ2.3: }\textit{What are the enabling and inhibiting factors influencing the transition to the Euro 2.0 digital currency}
\end{quoting}

\pagebreak

%--------------------CHAPTER 5-----------------
\section{Chapter 5} \label{sec:5}

\subsection{Introduction} \label{ssec:5:intro}

After explaining the features and implementation details of the Euro 2.0 system in \hypernameref{sec:3} and exploring the change in relationship with money of stakeholders \hypernameref{sec:4}, we are ready to analyze the risks of Euro 2.0 system. This chapter answers the research question:

\begin{quoting}
	\textbf{RQ3:} \textit{How do we assess the security, scalability, and privacy risks of Euro 2.0 digital currency system and potential impact on stakeholders?}
\end{quoting}

In which we take piece by piece in the research questions:
\begin{quoting}
	\textbf{RQ3.1: }\textit{What are risk criteria for the system consistent with the mission of the project?}
\end{quoting}
\begin{quoting}
	\textbf{RQ3.2: }\textit{What are the assets and information asset containers of the system?}
\end{quoting}
\begin{quoting}
	\textbf{RQ3.3: }\textit{What are the areas of concern and threats of the system?}
\end{quoting}
\begin{quoting}
	\textbf{RQ3.4: }\textit{What are the risks of the system and potential impact on stakeholders?}
\end{quoting}
\begin{quoting}
	\textbf{RQ3.5: }\textit{What are the mitigation strategies or alternative design suggestions for the Euro 2.0 system?}
\end{quoting}

\textit{RQ3.1} through \textit{RQ3.5} are answered by applying Steps 1 through 8 of the OCTAVE Allegro risk assessment methodology\cite{CaralliIntroducingOCTAVE2007} to the system described in \hypernameref{sec:3}. Section \ref{ssec:5.1} answers \textit{RQ3.1} by establishing a risk measurement criteria. Section \ref{ssec:5.2} works through constructing an information asset profile and identifying information asset containers to answer \textit{RQ3.2}. Section \ref{ssec:5.3} uses current research in blockchain security, scalability, and privacy to identify concerns and threats to answer \textit{RQ3.3}. In Section \ref{ssec:5.4}, found threats are used in risk identification and analysis to answer \textit{RQ3.4}. Section \ref{ssec:5.5} uses the top risks from of the risk analysis to propose mitigation strategies and Euro 2.0 design suggestions answering \textit{RQ3.5}. Finally from the above analysis we can make conclusions of the overal risks in Euro 2.0 to answer \textit{RQ3} in the Conclusion \ref{ssec:5.6}.

\subsection{Establishing Risk Criteria} \label{ssec:5.1}

\subsection{Identifying Assets and Containers} \label{ssec:5.2}

\subsection{Identifying Threats} \label{ssec:5.3}

\subsection{Risk Analysis} \label{ssec:5.4}

\subsection{Mitigation Approach} \label{ssec:5.5}

\subsection{Conclusion} \label{ssec:5.6}

\pagebreak

%--------------------CHAPTER 6 - EVALUATION-----------------
\section{Evaluation} \label{sec:6}

\todo[inline]{Write Evaluation (Chapter 6) \ldots}

\pagebreak

%-------------------------------SUMMARY---------------------------
\section{Summary} \label{sec:7}

\todo[inline]{Write Summary (Chapter 7) \ldots}

\pagebreak

%------------------------------Bibliography-----------------------------------
\addcontentsline{toc}{section}{References} \label{sec:refs}
\bibliography{thesis}{References}
\bibliographystyle{plain}
\pagebreak

%-----------------------------APPENDICES--------------------------------
\section*{Appendix A - Files} \label{sec:a}
%\label{Lisa1}
\addcontentsline{toc}{section}{Appendix A - Files}

\subsection*{A.1: Euro 2.0 Codebase} \label{ssec:a.1}
\url{http://github.com/cryptofiat}

\subsection*{A.2: Archive of Cited Web Pages} \label{ssec:a.2}

\subsection*{A.3: Euro 2.0 Contract Codebase} \label{ssec:a.3}
\url{https://github.com/cryptofiat/contract}

\subsection*{A.4: CryptoFiat Contract} \label{ssec:a.4}
\url{https://github.com/cryptofiat/contract/blob/master/CryptoFiat.sol}

\end{document}
