%
%  $Description: Author guidelines and sample document in LaTeX 2.09/2e$ 
%
%  $Author: Priit Ruberg$
%  $Date: 2015/02/09 $
%  $Revision: 2.5 $
%  
%  Translated to English 2017/03/10 by Chris Raastad
%
\documentclass[12pt]{article} %Document class definition and text size settings
%
%Packages can be explored in more details: https://www.ctan.org/pkg/PACKAGE_NAME?lang=en
%
\usepackage{graphicx} %Allow using graphics in the text
\usepackage[top=2.5cm, bottom=2.5cm, left=3cm, right=3cm]{geometry} %Set the page margins
\usepackage{titlesec} %Package for title style
\usepackage{longtable} %Package so tables can be longer than one page
\usepackage{multirow} %Package so table cells can span multiple rows
\usepackage{todonotes} %Package so you can add nice TODO marks in your paper with \todo{TODO text...}
\usepackage{url} %Package in order to nicely use URLs
\usepackage{float} %Package to improves interface for defining floating objects like figures and tables

\usepackage[english, russian, estonian]{babel} %Specifies possible languages of the document: English, Russian, and Estonian
	\addto\captionsestonian{\def\refname{\centerline{References}}} %Changes references name and makes it center
	\addto\captionsestonian{\def\listfigurename{\centerline{List of figures}}} %Changes drawing list name and makes it center
	\addto\captionsestonian{\def\listtablename{\centerline{List of tables}}} %Changes table list name makes it center
	\addto\captionsestonian{\def\contentsname{\centerline{Table of contents}}}
\usepackage[T2A,T1]{fontenc} %Font encodings for Russian and Estonian letters
\usepackage[utf8]{inputenc} %use UTF8 decodings

\usepackage{tocloft} %Control table of contents, tables, etc.
%\setlength\cftparskip{-2pt}
%\setlength\cftbeforechapskip{0pt}

\usepackage{amssymb} %For square itemized listss
\renewcommand{\labelitemi}{\tiny$\blacksquare$} %For square itemized lists


\usepackage{caption} %Needed to customise captions for tables and figures
\captionsetup{labelsep=period} %Set table and figure caption name to be separated with text with a period

\usepackage{verbatimbox} %To put program code in the center using Verbatim

\titlelabel{\thetitle.\quad} %Adds periods to the end of titles

\usepackage{times} %Sets font to Times New Roman
\usepackage{fancyhdr} %Allows more control of headers and footers
\setlength{\parindent}{0cm} %Set paragraph indentation to zero
\usepackage{setspace} %Allows setting spacing between lines
\onehalfspacing %Set spacing to 1.5x
%\usepackage{parskip}
\setlength{\parskip}{\baselineskip}
%\hangindent=0.7cm

\hyphenation{põhi-tekstis üliõpilas-kood lehe-küljed joonda-takse} %Correcting incorrect hyphenation (?)

\begin{document}

%------------------------------ENGLISH TITLE PAGE---------------------------------
\thispagestyle{fancy} %Page will include header and footer
\renewcommand{\headrulewidth}{0pt} %Remove header horizontal line
\renewcommand{\footrulewidth}{0pt} %Remove footer horizontal line
\headheight = 57pt %Set header heght (with regards to compiler suggestion)
\footskip = 11pt %Footer space
\headsep = 0pt %Decrease header and text line spacing distance to zero

\chead{ %Place the following text header to the center
 \textsc{\begin{Large} %Make the following text have big letters
	TALLINN UNIVERSITY OF TECHNOLOGY\\
	\end{Large}}
	Faculty of Information Technology\\
	Department of Computer Engineering
}
\vspace*{7 cm} %Make the page beginning and text line spacing correspond to the width

\begin{center} %Text centered
ITC70LT\\[0cm]
Christopher David Raastad\\
\begin{LARGE}
\textsc{Thesis title\\}  \todo[noline]{Write thesis title}
\end{LARGE}
Master thesis\\[2cm]
\end{center}

\begin{flushright} %Align text to the right
Alex Norta\\[0cm]
PhD\\[0cm]
Associated Professor\\[0cm]
\end{flushright}

\cfoot{Tallinn 2017} %Add location and year to the header
%\renewcommand{\headrulewidth}{0pt} %Remove the footer horizontal line
\pagebreak %End of page

%------------------------------TIITELLEHT EESTI KEELES---------------------------------
\thispagestyle{fancy} %Leht sisaldab päist ja jalust
\renewcommand{\headrulewidth}{0pt} %Eemaldab päisest horisontaalse joone
\renewcommand{\footrulewidth}{0pt} %Eemaldab jalusest horisontaalse joone
\headheight = 57pt %Paneb paika päise laiuse (vastavalt kompilaatori soovitusele)
\footskip = 11pt %Jaluse ruum
\headsep = 0pt %Vähendab päise ja teksti vahelise kauguse nullini

\chead{ %Paigutab järgneva teksti päises keskele
 \textsc{\begin{Large} %Tekst suurtähtedega ja suuremaks
	tallinna tehnikaülikool\\
	\end{Large}}
	Infotehnoloogia teaduskond\\
	Arvutitehnika instituut
}
\vspace*{7 cm} %Tekitab lehe alguse ja teksti vahele tühja ala vastava laiusega

\begin{center} %Tekst keskele
ITC70LT\\[0cm]
Christopher David Raastad\\
\begin{LARGE}
\textsc{lõputöö pealkiri\\}  \todo[noline]{Kirjuta pealkirja eesti keeles}
\end{LARGE}
Magister\\[2cm]
\end{center}

\begin{flushright} %Joondab teksti paremale
Alex Norta\\[0cm]  \todo[noline]{Tõlgi PhD eesti keelde}
PhD\\[0cm]
Associated Professor\\[0cm]  \todo[noline]{Tõlgi Associated Professor eesti keelde}
\end{flushright}

\cfoot{Tallinn 2015} %Lisab asukoha ja kuupäeva jalusesse
%\renewcommand{\headrulewidth}{0pt} %Eemaldab päisest horisontaalse joone
\pagebreak %Lehe lõpp


%----------------------------LIST OF TODOS----------------------------------
\listoftodos
\newpage

%---------------------------AUTHOR DECLARATION-------------------------
\section*{\begin{center}
 Author’s declaration of originality
\end{center}}
I hereby certify that I am the sole author of this thesis. All the used materials, references to the literature and the work of others have been referred to. This thesis has not been presented for examination anywhere else.

Author: Christopher David Raastad

May 8th 2017
\pagebreak

%---------------------------ABSTRACT---------------------------------
\section*{\begin{center}
Abstract
\end{center}}

\todo[inline]{Write English abstract \ldots}

If the thesis is written in English, the abstract is $\frac{1}{2}$ A4 long and the abstract in Estonian (\textit{Annotatsioon}) is of length 1 A4.

The last paragraph of abstract is obligatory and must be written accordingly:

\todo[inline]{Fill in English abstract thesis details \ldots}

The thesis is in English and contains [pages] pages of text, [chapters] chapters, [figures] figures, [tables] tables.

\pagebreak

%-----------------------------ANNOTATSIOON-----------------------------------
\section*{\begin{center}
Annotatsioon
\end{center}}

\todo[inline]{Kirjuta annotatsiooni eesti keeles \ldots}

Kui töö põhikeel on inglise keel, siis esitatakse annotatsioon (Abstract) inglise keeles mahuga $\frac{1}{2}$ A4 lehekülge ja annotatsioon eesti keeles mahuga vähemalt 1 A4 lehekülg.

Annotatsiooni viimane lõik on kohustuslik ja omab järgmist sõnastust:

\todo[inline]{Täitke eesti keele annotatsiooni lõputöö detailid \ldots}

Lõputöö on kirjutatud [mis keeles] keeles ning sisaldab teksti [lehekülgede arv] leheküljel, [peatükkide arv] peatükki, [jooniste arv] joonist, [tabelite arv] tabelit.

\pagebreak

%---------------------ABBREVIATIONS AND GLOSSARY OF TERMS---------------------
\section*{\begin{center}
Table of abbreviations and terms
\end{center}}


\begin{tabular}{p{3 cm}ll} %Table where the first cell width is 3cm
TTÜ & \textit{Tallinna Tehnikal Üllikool}, Tallinn University of Technology\\
ATI & TTÜ \textit{Arvutitehnika instituut}, Department of Computer Science\\
DPI & Dots per inch
\end{tabular}

\todo[inline]{Continue adding to table of abbreviations and delete old ones\ldots}

\pagebreak

%----------------------------TABLE OF CONTENTS----------------------------------
\tableofcontents
\newpage

%----------------------LIST OF DRAWINGS-------------------------------
\listoffigures
\pagebreak

%----------------------LIST OF TABLES---------------------------------
\listoftables
\pagebreak

%-----------------------------CHAPTER 1 - INTRODUCTION------------------------------- 
\section{Introduction}
\label{Introduction} %Allows you to refer to the title with the \ref{Introduction} command

\todo[inline]{Write the Introduction (Chapter 1) \ldots}

\pagebreak

%--------------------CHAPTER 2 - BRIDGE-----------------
\section{Bridge of Knowledge}
\label{Bridge of Knowledge}

\todo[inline]{Write the Bridge of Knowledge (Chapter 2) \ldots}

\pagebreak

%--------------------CHAPTER 3-----------------
\section{Chapter 3}
\label{Chapter 3}

\todo[inline]{Write Chapter 3 \ldots}

\pagebreak

%--------------------CHAPTER 4-----------------
\section{Chapter 4}
\label{Chapter 4}

\todo[inline]{Write Chapter 4 \ldots}

\pagebreak

%--------------------CHAPTER 5-----------------
\section{Chapter 5}
\label{Chapter 5}

\todo[inline]{Write Chapter 5 \ldots}

\pagebreak

%--------------------CHAPTER 6 - EVALUATION-----------------
\section{Evaluation}
\label{Evaluation}

\todo[inline]{Write Evaluation (Chapter 6) \ldots}

\pagebreak

%-------------------------------SUMMARY---------------------------
\section{Summary}
\label{Summary}

\todo[inline]{Write Summary (Chapter 7) \ldots}

\pagebreak

%------------------------------Bibliography-----------------------------------
\addcontentsline{toc}{section}{Bibliography}

\todo[inline]{Move these references to Biblio.bib}

\begin{thebibliography} {9}
\bibitem{l6petajale13}
Arvutitehnika instituut. Lõpetajale. [WWW] http://ati.ttu.ee/index.php?page=470 (13.05.2013)
\end{thebibliography}

Kasutatud kirjanduse loetelu koostamise näidet vaata TTÜ Raamatukogu juhendmaterjalist „Viitekirjete koostamine“ aadressil: \texttt{http://www.ttu.ee/public/r/raamatukogu/juhendid/
viitekirjetekoostamine.pdf}.

\pagebreak

%-----------------------------APPENDICES--------------------------------
\section*{Appendix 1 - [Heading]}
%\label{Lisa1}
\addcontentsline{toc}{section}{Appendix 1}

\todo[inline]{Add Appendix 1 or delete it}

\end{document}
