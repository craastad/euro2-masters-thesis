%
%  $Description: Author guidelines and sample document in LaTeX 2.09/2e$ 
%
%  $Author: Priit Ruberg$
%  $Date: 2015/02/09 $
%  $Revision: 2.5 $
%  
%  Translated to English 2017/03/10 by Chris Raastad
%
\documentclass[12pt]{article} %Document class definition and text size settings
%
%Packages can be explored in more details: https://www.ctan.org/pkg/PACKAGE_NAME?lang=en
%
\usepackage{graphicx} %Allow using graphics in the text
\usepackage[top=2.5cm, bottom=2.5cm, left=3cm, right=3cm]{geometry} %Set the page margins
\usepackage{titlesec} %Package for title style
\usepackage{longtable} %Package so tables can be longer than one page
\usepackage{multirow} %Package so table cells can span multiple rows
\usepackage[colorinlistoftodos]{todonotes} %Package so you can add nice TODO marks in your paper with \todo{TODO text...}
\usepackage{cite} %For Bibtex

\usepackage[hidelinks]{hyperref}
\usepackage{url} %Package in order to nicely use URLs
\usepackage{float} %Package to improves interface for defining floating objects like figures and tables

\usepackage[english, estonian]{babel} %Specifies possible languages of the document: English, Russian, and Estonian
	\addto\captionsestonian{\def\refname{\centerline{References}}} %Changes references name and makes it center
	\addto\captionsestonian{\def\listfigurename{\centerline{List of figures}}} %Changes drawing list name and makes it center
	\addto\captionsestonian{\def\listtablename{\centerline{List of tables}}} %Changes table list name makes it center
	\addto\captionsestonian{\def\contentsname{\centerline{Table of contents}}}
\usepackage[T2A,T1]{fontenc} %Font encodings for Russian and Estonian letters
\usepackage[utf8]{inputenc} %use UTF8 decodings

\usepackage{tocloft} %Control table of contents, tables, etc.
%\setlength\cftparskip{-2pt}
%\setlength\cftbeforechapskip{0pt}

\usepackage{amssymb} %For square itemized listss
\renewcommand{\labelitemi}{\tiny$\blacksquare$} %For square itemized lists


\usepackage{caption} %Needed to customise captions for tables and figures
\captionsetup{labelsep=period} %Set table and figure caption name to be separated with text with a period

\usepackage{verbatimbox} %To put program code in the center using Verbatim

\titlelabel{\thetitle.\quad} %Adds periods to the end of titles

\usepackage{times} %Sets font to Times New Roman
\usepackage{fancyhdr} %Allows more control of headers and footers
\setlength{\parindent}{0cm} %Set paragraph indentation to zero
\usepackage{setspace} %Allows setting spacing between lines
\onehalfspacing %Set spacing to 1.5x
%\usepackage{parskip}
\setlength{\parskip}{\baselineskip}
%\hangindent=0.7cm

\hyphenation{põhi-tekstis üliõpilas-kood lehe-küljed joonda-takse} %Correcting incorrect hyphenation (?)

\begin{document}

%------------------------------ENGLISH TITLE PAGE---------------------------------
\thispagestyle{fancy} %Page will include header and footer
\renewcommand{\headrulewidth}{0pt} %Remove header horizontal line
\renewcommand{\footrulewidth}{0pt} %Remove footer horizontal line
\headheight = 57pt %Set header heght (with regards to compiler suggestion)
\footskip = 11pt %Footer space
\headsep = 0pt %Decrease header and text line spacing distance to zero

\chead{ %Place the following text header to the center
 \textsc{\begin{Large} %Make the following text have big letters
	TALLINN UNIVERSITY OF TECHNOLOGY\\
	\end{Large}}
	Faculty of Information Technology\\
	Department of Computer Engineering
}
\vspace*{7 cm} %Make the page beginning and text line spacing correspond to the width

\begin{center} %Text centered
ITC70LT\\[0cm]
Christopher David Raastad\\
\begin{LARGE}
\textsc{Thesis title\\}  \todo[noline]{Write thesis title}
\end{LARGE}
Master thesis\\[2cm]
\end{center}

\begin{flushright} %Align text to the right
Alex Norta\\[0cm]
PhD\\[0cm]
Associated Professor\\[0cm]
\end{flushright}

\cfoot{Tallinn 2017} %Add location and year to the header
%\renewcommand{\headrulewidth}{0pt} %Remove the footer horizontal line
\pagebreak %End of page

%------------------------------TIITELLEHT EESTI KEELES---------------------------------
\thispagestyle{fancy} %Leht sisaldab päist ja jalust
\renewcommand{\headrulewidth}{0pt} %Eemaldab päisest horisontaalse joone
\renewcommand{\footrulewidth}{0pt} %Eemaldab jalusest horisontaalse joone
\headheight = 57pt %Paneb paika päise laiuse (vastavalt kompilaatori soovitusele)
\footskip = 11pt %Jaluse ruum
\headsep = 0pt %Vähendab päise ja teksti vahelise kauguse nullini

\chead{ %Paigutab järgneva teksti päises keskele
 \textsc{\begin{Large} %Tekst suurtähtedega ja suuremaks
	tallinna tehnikaülikool\\
	\end{Large}}
	Infotehnoloogia teaduskond\\
	Arvutitehnika instituut
}
\vspace*{7 cm} %Tekitab lehe alguse ja teksti vahele tühja ala vastava laiusega

\begin{center} %Tekst keskele
ITC70LT\\[0cm]
Christopher David Raastad\\
\begin{LARGE}
\textsc{lõputöö pealkiri\\}  \todo[noline]{Kirjuta pealkirja eesti keeles}
\end{LARGE}
Magister\\[2cm]
\end{center}

\begin{flushright} %Joondab teksti paremale
Alex Norta\\[0cm]  \todo[noline]{Tõlgi PhD eesti keelde}
PhD\\[0cm]
Associated Professor\\[0cm]  \todo[noline]{Tõlgi Associated Professor eesti keelde}
\end{flushright}

\cfoot{Tallinn 2015} %Lisab asukoha ja kuupäeva jalusesse
%\renewcommand{\headrulewidth}{0pt} %Eemaldab päisest horisontaalse joone
\pagebreak %Lehe lõpp


%----------------------------LIST OF TODOS----------------------------------
\listoftodos
\newpage

%---------------------------AUTHOR DECLARATION-------------------------
\section*{\begin{center}
 Author’s declaration of originality
\end{center}}
I hereby certify that I am the sole author of this thesis. All the used materials, references to the literature and the work of others have been referred to. This thesis has not been presented for examination anywhere else.

Author: Christopher David Raastad

May 8th 2017
\pagebreak

%---------------------------ABSTRACT---------------------------------
\section*{\begin{center}
Abstract
\end{center}}

\todo[inline]{Write English abstract \ldots}

If the thesis is written in English, the abstract is $\frac{1}{2}$ A4 long and the abstract in Estonian (\textit{Annotatsioon}) is of length 1 A4.

The last paragraph of abstract is obligatory and must be written accordingly:

\todo[inline]{Fill in English abstract thesis details \ldots}

The thesis is in English and contains [pages] pages of text, [chapters] chapters, [figures] figures, [tables] tables.

\pagebreak

%-----------------------------ANNOTATSIOON-----------------------------------
\section*{\begin{center}
Annotatsioon
\end{center}}

\todo[inline]{Kirjuta annotatsiooni eesti keeles \ldots}

Kui töö põhikeel on inglise keel, siis esitatakse annotatsioon (Abstract) inglise keeles mahuga $\frac{1}{2}$ A4 lehekülge ja annotatsioon eesti keeles mahuga vähemalt 1 A4 lehekülg.

Annotatsiooni viimane lõik on kohustuslik ja omab järgmist sõnastust:

\todo[inline]{Täitke eesti keele annotatsiooni lõputöö detailid \ldots}

Lõputöö on kirjutatud [mis keeles] keeles ning sisaldab teksti [lehekülgede arv] leheküljel, [peatükkide arv] peatükki, [jooniste arv] joonist, [tabelite arv] tabelit.

\pagebreak

%---------------------ABBREVIATIONS AND GLOSSARY OF TERMS---------------------
\section*{\begin{center}
Table of abbreviations and terms
\end{center}}


\begin{tabular}{p{3 cm}ll} %Table where the first cell width is 3cm
BTC & Bitcoin currency code
\end{tabular}

\todo[inline]{Continue adding to table of abbreviations and delete old ones\ldots}

\pagebreak

%----------------------------TABLE OF CONTENTS----------------------------------
\tableofcontents
\newpage

%----------------------LIST OF DRAWINGS-------------------------------
\listoffigures
\pagebreak

%----------------------LIST OF TABLES---------------------------------
\listoftables
\pagebreak

%-----------------------------CHAPTER 1 - INTRODUCTION------------------------------- 
\section{Introduction}
\label{Introduction}

\subsection{Setting the Stage}
Payments today are the laggard of the information age. While emails can be sent instantly money is either slow and/or expensive to move digitally. Bank transfers can take days, only work during business hours, and have high fees across borders. Card payments are instant but enslave merchants with high fees and chargeback risks. Paypal brought payments to the internet, but still brings a cost to accept payments and moving funds across borders. Fintech companies like Venmo and Square can make the illusion of fast payments, but still take days to settle in the background with the same chargeback risks. All of this inconvenience comes from the centralised Financial system of banking that has little economic motivation to innovate and undo legacy. A usable digital currency would greatly alleviate this friction in transferring value that costs the economy an estimated 1\% of GDP annually \cite{kaarmann2013cost}.

Bitcoin was the first mover in digital currency introducing a clever mechanism of digital value transfer completely sidestepping the existing financial system\cite{nakamoto2008bitcoin}. Its distributed proof of work consensus protocol, clever economical incentives to maintain the network, irreversible transactions, and pseudo anonymous users sent waves of interest and skepticism in the financial and regulatory community. At the beginning of 2014, 5 years after the genesis block, the world's first cryptocurrency already had billions of dollars market capitalisation\cite{coinmarketcap2017bitcoin}, more than 100,000 daily active users doubling roughly every 8 months\cite{RePEc:fip:fedgfe:2014-104}, and hundreds of companies founded to drive the ecosystem\todo{citation needed for number of companies}. Altcoins forked the Bitcoin blockchain technology to tackle other use cases and attempt to overcome shortcomings of Bitcoin. Exchanges sprung up to bring institutional trust into the ecosystem, making it easier to buy and sell bitcoins and bridging the gap between the traditional world of finance and digital currency. Eventually Bitcoin could be used to buy Domino's pizza and airline tickets\todo{citation needed for dominos pizza and airline tickets}.

Yet payments are a niche use case in Bitcoin and other cryptocurrencies\cite{sas2016design}. The main use case thus far being long term value storage investment and short term ``get-rich-quick'' speculation. Bitcoin is complicated to use, takes minutes to confidently clear transactions, lacks clear governance structure, has long term scaling worries, and missing key features of trust and user identity making it unfavourable for mainstream commerce. In addition converting the digital currency to and from the banking financial world comes at a cost. Providers of goods and services can ``accept'' Bitcoin but in reality immediately convert the funds to a fiat currency at a 1\% fee\todo{citation for 1\% transaction fee}. The real world is not yet priced in BTC. Bitcoin does not solve the problems of the majority of society running on the traditional financial system.

The solution is to bring fiat currency and the traditional financial system into the realm of digital currency. This manuscript explores Euro 2.0 digital currency system. Trust, regulation, and convenience are built in with Estonian ID and the Ethereum blockchain smart contract technology. The system removes the need of financial institution intermediaries to profit off of holding balances and executing payments. The usability and trust for payments arises from ease of sending to personal Estonian ID codes, identification of users on the system, and use of a common fiat currency, Euro. We explore how to derive the need of this system for stakeholders, the technical requirements, and analyse the security and privacy of its implementation. The system can be initially managed by a foundation and completely run later by central banks saving the economy a majority of its 1\% GDP lost annually to payment friction \cite{kaarmann2013cost}.

\subsection{History of Currency and Payments}
\todo[inline]{Citations really needed \ldots}
Money and payments haven been part of human society since the dawn of civilisation. Currency grew out of the need to transport value of goods without transferring the goods themselves. First came gold, silver, and other precious metals to trade in exchange for goods and services. Later in the 1600s came notes issued by the Bank of England backed by silver and shortly after followed every country in the Western world. Over this time Banks grew as the de facto institutions store and transfer value in government currencies. By the 1950s the USA was the first government to eliminate a physical backing (silver) to create a fiat currency only backed by the trust of the US Government and Federal Reserve. Again, shortly after other countries followed suit and there are no countries left physically backing notes; money is completely a trust contract with the government and society.

As early as the 1960s banks were some of the first adopters of information technology, using computers and databases to overhaul paper processes. The electronic access of money was the Credit Card, its earliest niche private usage was the USA in the 1920s, Diners’ Club, Inc. in 1950, American Express Travel and Entertainment card in 1958, and finally the founding of VISA in 1976 to spread the usage globally\cite{britannica2016creditcard}. Debit cards directly debiting bank accounts hit market as early as 1966 in the USA, long before mass consumer adoption of technology, gaining popularity in the 80s and 90s with the rise of ATM network and merchant acceptance\cite{collins2011debitcard}. With all of these developments, the Financial institutions of banks and payment processors are the heart of system and source of high fees. Credit Cards cost  merchants 1-3\% transaction fees and carry the risk of costly chargebacks at the benefit of consumer convenience. Debit cards are slightly better, charging on average \$0.40 per swipe. \todo{Cite source of fees and average fees either for a country or the whole world}

\todo[inline]{Citations needed for average cost of bank transfers globally}
With consumer adoption of the internet came personal windows into our finances via online banking. Bank transfers can take days to settle and only work during working hours of weekdays. International bank transfers can take even longer and with a heavy 3-5\% fee.  Many transactions in financial systems, such as stock trades, can happen ``instantly'' but in fact take days to settle on the backend due to legacy paper based processes. \todo{citation needed for stock trades settlement times} The financial industry and has only in this time created pretty facades to their inefficient processes. Paypal succeeding to bring payments to the internet, but is still atrociously expensive to send and receive card payments, 2.9\% + \$0.30 USD plus a 2.5\% spread for international movement of funds\cite{paypal2017fees}.

Why all the friction? The economic cost of payments is an estimated at 1\% of GDP annually \cite{kaarmann2013cost}. Financial institutions designed themselves to profit from creating friction in payments. International money transfer moves between four or more levels of banks, each taking a small slice of the fee, until funds reach the destination bank \todo{cite cost and process of international money transfer}. Consumer banking creates the illusion of free domestic transfers though subsidisation, but makes the meat of their profits from transactions data to sell customers credit and loans. \todo {bank local transfer subsidisation and methods of making money} Institutional banks gamble with our money every day on Wall Street, when things get out of hand, we have the Financial meltdown of 2008.\todo{cite financial crisis of 2008} Even post financial crisis, too big to fail banks have barely changed there ways. With increased regulatory costs and greedy shareholders comes even less economical incentive to reduce the cost of payments.\todo{cite something referring to big banks being greedy or having no motivation to reduce fees}

\subsection{Bitcoin Revolution}
\todo[inline]{Citations needed \ldots}
Following an endless spiral of legacy infrastructure and greedy direction of banks of 2008, Satoshi Nakamoto quietly released \textit{Bitcoin: A Peer-to-Peer Electronic Cash System}\cite{nakamoto2008bitcoin}. The clever decentralised, proof of work, consensus system of value transfer hints at solving the business problem ``how do I create a system where nobody can stop me spending my own money?''\cite{brown2016introducing}. A by product of the the Bitcoin system is the Blockchain distributed ledger technology. This consensus mechanism inspired developers to create hundreds of different altcoins to overcome some shortcomings of the Bitcoin and tackle a variety of uses cases with similar technology.

\subsection{Altcoins and Exchanges}
\todo[inline]{Citations needed \ldots}
Bitcoin and altcoins have until now had more traction for long term investment and get-rich-quick speculation then becoming the next generation payment method. Websites and stores accepting Bitcoin do so more for marketing then actual economic incentive. Accepting Bitcoin with a payment provider is in fact just charging a 1\% fee to receive payment directly to a fiat currency bank account with development costs of integrating a new payment method.

\subsection{What's Missing for Digital Currency?}
\todo[inline]{Citations needed Bitcoin HCI study \ldots}
Despite this promising rise of the value transfer protocol, payments for goods and services are still a very niche use case for Bitcoin and other cryptocurrencies. What's preventing cryptocurrencies from widespread adoption for payments? Ironically the defining features of Bitcoin, pseudo anonymity, irreversibility of transactions, and lack of central regulations, make for an unattractive payments system. Exchanges arose to be a more trustworthy source to buy and sell Bitcoins and create a regulated gateway to the traditional financial system. Transacting with individuals outside of exchanges poses great risk with off chain components of transactions. Users end up doing KYC on their counter-party or rely on other community verification to gain trust in completing transactions.

\subsection{Research Questions}

\subsection{Diving into Euro 2.0}

\pagebreak

%--------------------CHAPTER 2 - BRIDGE-----------------
\section{Bridge of Knowledge}
\label{Bridge of Knowledge}

\todo[inline]{Write the Bridge of Knowledge (Chapter 2) \ldots}

\pagebreak

%--------------------CHAPTER 3-----------------
\section{Chapter 3}
\label{Chapter 3}

\todo[inline]{Write Chapter 3 \ldots}

\pagebreak

%--------------------CHAPTER 4-----------------
\section{Chapter 4}
\label{Chapter 4}

\todo[inline]{Write Chapter 4 \ldots}

\pagebreak

%--------------------CHAPTER 5-----------------
\section{Chapter 5}
\label{Chapter 5}

\todo[inline]{Write Chapter 5 \ldots}

\pagebreak

%--------------------CHAPTER 6 - EVALUATION-----------------
\section{Evaluation}
\label{Evaluation}

\todo[inline]{Write Evaluation (Chapter 6) \ldots}

\pagebreak

%-------------------------------SUMMARY---------------------------
\section{Summary}
\label{Summary}

\todo[inline]{Write Summary (Chapter 7) \ldots}

\pagebreak

%------------------------------Bibliography-----------------------------------
\addcontentsline{toc}{section}{Bibliography}
\bibliography{thesis}{}
\bibliographystyle{plain}
\pagebreak

%-----------------------------APPENDICES--------------------------------
\section*{Appendix 1 - [Heading]}
%\label{Lisa1}
\addcontentsline{toc}{section}{Appendix 1}

\todo[inline]{Add Appendix 1 or delete it}

\end{document}
