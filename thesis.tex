%
%  $Description: Author guidelines and sample document in LaTeX 2.09/2e$ 
%
%  $Author: Priit Ruberg$
%  $Date: 2015/02/09 $
%  $Revision: 2.5 $
%  
%  Translated to English 2017/03/10 by Chris Raastad
%
\documentclass[12pt]{article} %Document class definition and text size settings
%
%Packages can be explored in more details: https://www.ctan.org/pkg/PACKAGE_NAME?lang=en
%
\usepackage{graphicx} %Allow using graphics in the text
\usepackage[top=2.5cm, bottom=2.5cm, left=3cm, right=3cm]{geometry} %Set the page margins
\usepackage{titlesec} %Package for title style
\usepackage{longtable} %Package so tables can be longer than one page
\usepackage{multirow} %Package so table cells can span multiple rows
\usepackage[colorinlistoftodos]{todonotes} %Package so you can add nice TODO marks in your paper with \todo{TODO text...}
\usepackage{cite} %For Bibtex

\usepackage{hyperref}
% \usepackage{url} %Package in order to nicely use URLs
\usepackage{float} %Package to improves interface for defining floating objects like figures and tables

\usepackage[english, estonian]{babel} %Specifies possible languages of the document: English, Russian, and Estonian
	\addto\captionsestonian{\def\refname{\centerline{References}}} %Changes references name and makes it center
	\addto\captionsestonian{\def\listfigurename{\centerline{List of figures}}} %Changes drawing list name and makes it center
	\addto\captionsestonian{\def\listtablename{\centerline{List of tables}}} %Changes table list name makes it center
	\addto\captionsestonian{\def\contentsname{\centerline{Table of contents}}}
\usepackage[T2A,T1]{fontenc} %Font encodings for Russian and Estonian letters
\usepackage[utf8]{inputenc} %use UTF8 decodings

\usepackage{tocloft} %Control table of contents, tables, etc.
%\setlength\cftparskip{-2pt}
%\setlength\cftbeforechapskip{0pt}

\usepackage{amssymb} %For square itemized listss
\renewcommand{\labelitemi}{\tiny$\blacksquare$} %For square itemized lists


\usepackage{caption} %Needed to customise captions for tables and figures
\captionsetup{labelsep=period} %Set table and figure caption name to be separated with text with a period

\usepackage{verbatimbox} %To put program code in the center using Verbatim

\titlelabel{\thetitle.\quad} %Adds periods to the end of titles

\usepackage{times} %Sets font to Times New Roman
\usepackage{fancyhdr} %Allows more control of headers and footers
\setlength{\parindent}{0cm} %Set paragraph indentation to zero
\usepackage{setspace} %Allows setting spacing between lines
\onehalfspacing %Set spacing to 1.5x
%\usepackage{parskip}
\setlength{\parskip}{\baselineskip}
%\hangindent=0.7cm

\hyphenation{põhi-tekstis üliõpilas-kood lehe-küljed joonda-takse} %Correcting incorrect hyphenation (?)

\usepackage{eurosym}
\usepackage{quoting} %for quotation format
\quotingsetup{vskip=0pt,leftmargin=0pt} %for no whitespace around quotations

\begin{document}

%------------------------------ENGLISH TITLE PAGE---------------------------------
\thispagestyle{fancy} %Page will include header and footer
\renewcommand{\headrulewidth}{0pt} %Remove header horizontal line
\renewcommand{\footrulewidth}{0pt} %Remove footer horizontal line
\headheight = 57pt %Set header heght (with regards to compiler suggestion)
\footskip = 11pt %Footer space
\headsep = 0pt %Decrease header and text line spacing distance to zero

\chead{ %Place the following text header to the center
 \textsc{\begin{Large} %Make the following text have big letters
	TALLINN UNIVERSITY OF TECHNOLOGY\\
	\end{Large}}
	Faculty of Information Technology\\
	Department of Computer Engineering
}
\vspace*{7 cm} %Make the page beginning and text line spacing correspond to the width

\begin{center} %Text centered
ITC70LT\\[0cm]
Christopher David Raastad\\
\begin{LARGE}
\textsc{Security, Scalability, and Privacy Analysis of the Euro 2.0 Digital Currency\\}
\end{LARGE}
Master thesis\\[2cm]
\end{center}

\begin{flushright} %Align text to the right
Alex Norta\\[0cm]
PhD\\[0cm]
Associated Professor\\[0cm]
\end{flushright}

\cfoot{Tallinn 2017} %Add location and year to the header
%\renewcommand{\headrulewidth}{0pt} %Remove the footer horizontal line
\pagebreak %End of page

%------------------------------TIITELLEHT EESTI KEELES---------------------------------
\thispagestyle{fancy} %Leht sisaldab päist ja jalust
\renewcommand{\headrulewidth}{0pt} %Eemaldab päisest horisontaalse joone
\renewcommand{\footrulewidth}{0pt} %Eemaldab jalusest horisontaalse joone
\headheight = 57pt %Paneb paika päise laiuse (vastavalt kompilaatori soovitusele)
\footskip = 11pt %Jaluse ruum
\headsep = 0pt %Vähendab päise ja teksti vahelise kauguse nullini

\chead{ %Paigutab järgneva teksti päises keskele
 \textsc{\begin{Large} %Tekst suurtähtedega ja suuremaks
	tallinna tehnikaülikool\\
	\end{Large}}
	Infotehnoloogia teaduskond\\
	Arvutitehnika instituut
}
\vspace*{7 cm} %Tekitab lehe alguse ja teksti vahele tühja ala vastava laiusega

\begin{center} %Tekst keskele
ITC70LT\\[0cm]
Christopher David Raastad\\
\begin{LARGE}
\textsc{Euro 2.0 Digitaalne Valuuta Turvalisuse, Skaleeritamise, ja Privaatsuse Analüüs\\}
\end{LARGE}
Magister\\[2cm]
\end{center}

\begin{flushright} %Joondab teksti paremale
Alex Norta\\[0cm]  \todo[noline]{Tõlgi PhD eesti keelde}
PhD\\[0cm]
Associate Professor\\[0cm]  \todo[noline]{Tõlgi Associated Professor eesti keelde}
\end{flushright}

\cfoot{Tallinn 2015} %Lisab asukoha ja kuupäeva jalusesse
%\renewcommand{\headrulewidth}{0pt} %Eemaldab päisest horisontaalse joone
\pagebreak %Lehe lõpp


%----------------------------LIST OF TODOS----------------------------------
\listoftodos
\newpage

%---------------------------AUTHOR DECLARATION-------------------------
\section*{\begin{center}
 Author’s declaration of originality
\end{center}}
I hereby certify that I am the sole author of this thesis. All the used materials, references to the literature and the work of others have been referred to. This thesis has not been presented for examination anywhere else.

Author: Christopher David Raastad

May 8th 2017
\pagebreak

%---------------------------ABSTRACT---------------------------------
\section*{\begin{center}
Abstract
\end{center}}

\todo[inline]{Write English abstract \ldots}

If the thesis is written in English, the abstract is $\frac{1}{2}$ A4 long and the abstract in Estonian (\textit{Annotatsioon}) is of length 1 A4.

The last paragraph of abstract is obligatory and must be written accordingly:

\todo[inline]{Fill in English abstract thesis details \ldots}

The thesis is in English and contains [pages] pages of text, [chapters] chapters, [figures] figures, [tables] tables.

\pagebreak

%-----------------------------ANNOTATSIOON-----------------------------------
\section*{\begin{center}
Annotatsioon
\end{center}}

\todo[inline]{Kirjuta annotatsiooni eesti keeles \ldots}

Kui töö põhikeel on inglise keel, siis esitatakse annotatsioon (Abstract) inglise keeles mahuga $\frac{1}{2}$ A4 lehekülge ja annotatsioon eesti keeles mahuga vähemalt 1 A4 lehekülg.

Annotatsiooni viimane lõik on kohustuslik ja omab järgmist sõnastust:

\todo[inline]{Täitke eesti keele annotatsiooni lõputöö detailid \ldots}

Lõputöö on kirjutatud [mis keeles] keeles ning sisaldab teksti [lehekülgede arv] leheküljel, [peatükkide arv] peatükki, [jooniste arv] joonist, [tabelite arv] tabelit.

\pagebreak

%---------------------ABBREVIATIONS AND GLOSSARY OF TERMS---------------------
\section*{\begin{center}
Table of abbreviations and terms
\end{center}}


\begin{tabular}{p{3 cm}ll} %Table where the first cell width is 3cm
BTC & Bitcoin currency code\\
AML & \textit{Anti Money Laundering}, processes implemented by Financial Institutions to hinder money laundering and comply with regulation\\
CTF & \textit{Counter Terrorist Financing}, processes implemented by Financial Institutions to hinder terrorist financing and comply with regulation\\
KYC & \textit{Know Your Customer}, proccesses of Financial institution to identify and verify customers' identities to aid AML and CTF
\end{tabular}

\todo[inline]{Continue adding to table of abbreviations and delete old ones\ldots}

\pagebreak

%----------------------------TABLE OF CONTENTS----------------------------------
\tableofcontents
\newpage

%----------------------LIST OF DRAWINGS-------------------------------
\listoffigures
\pagebreak

%----------------------LIST OF TABLES---------------------------------
\listoftables
\pagebreak

%-----------------------------CHAPTER 1 - INTRODUCTION------------------------------- 
\section{Introduction} \label{sec:1}

Payments today are the laggard of the information age. While emails can be sent instantly for free anywhere in the world, money is either slow and/or expensive to move digitally. Bank transfers can take days, only work during business hours, and have high fees across borders. Card payments are instant, but subject merchants to high fees and chargeback risks. Paypal brought convenient payments to the web, but at a high cost to accept payments and move funds internationally. Fintech companies like Venmo and Square can create the illusion of fast payments, but still take days to settle in the background with the potential chargeback risks. This cost comes from the highly regulated centralised financial payments systems with little economic incentive to reduce fees. A usable digital currency would greatly alleviate this friction in transferring value that theoretically costs the economy an estimated 1\% of GDP annually \cite{kaarmann2013cost}.

Bitcoin was the first successful implementation of a decentralised digital currency. Nakamoto's peer to peer digital cash system completely sidestepped the existing financial system\cite{nakamoto2008bitcoin}. Its clever proof of work consensus protocol, economic incentives for nodes to maintain the network, pseudo anonymity, and irreversible transactions implemented on the public ledger Blockchain technology has been called ``the next technological revolution''\todo{citation, probably from Blockchain Revolution}. Hundreds of Altcoins followed Bitcoin's lead borrowing from the Blockchain technology to implement other flavours of cryptocurrency and digital asset management. Despite the rapid growth of the cryptocurrency ecosystem, everyday payments are a niche use case for digital currencies, which are more commonly used for investment and financial speculation\cite{Khairuddin:2016:EMB:2851581.2892500}. Merchants accepting Bitcoin payments are in fact immediately converting funds to a fiat currency for a non negligible fee. Bitcoin Exchanges are high cost gatekeepers between the cryptocurrency and financial systems.

What is missing today are incentives for parties on both sides of payments to hold assets end to end in digital currency. The cost and friction of moving between two financial systems, crypto and fiat, eliminates the perceived benefits of crypto currencies. The Euro 2.0 project aims to implement fiat currency on digital currency technology and reduce friction by eliminating unnecessary financial intermediaries. The unanimous adoption of Estonian ID card in Estonia and the Ethereum distributed application platform make a prototype system possible. This thesis introduces the Euro 2.0 system, its impact on stakeholders in the payments, and assesses the security, scalability, and privacy risks of the implementation to determine whether this system can in fact reasonably support a national digital currency.

\subsection{Existing Body of Knowledge}
\todo[inline]{Existing body of knowledge to gap detection}


\subsection{Research Questions}

This thesis explores the feasibility of the Euro 2.0 digital currency by answering the following research question:
\begin{quotation}
	\textbf{RQ: }\textit{How do we assure that the proposed Euro 2.0 digital currency does not have any major security, scalability, or privacy flaws compromising the usefulness of the monetary system?}
\end{quotation}

In which we explore in sub research questions:
\begin{quotation}
\textbf{RQ1: }\textit{How do we describe the relevant features needed for analysis of the Euro 2.0 digital currency?}
\end{quotation}
\begin{quotation}
\textbf{RQ2: }\textit{How does adoption of Euro 2.0 digital currency impact the relationship with money for key stakeholders: users, merchants, governments, and financial institutions?}
\end{quotation}
\begin{quotation}
\textbf{RQ3: }\textit{How do we assess the security, scalability, and privacy risks of Euro 2.0 digital currency adoption and potential impact on stakeholders?}
\end{quotation}

Answering \textbf{RQ} will result in one of two possible outcomes:
\begin{itemize}
	\item The Euro 2.0 system is free from significant risks in functioning as a digital monetary system.
	\item The Euro 2.0 system has one or more serious flaws that need to be adressed before significant adoption.
\end{itemize}
In order to do a risk analysis, the key features of the Euro 2.0 system need to be identified and implementation details described by answering \textbf{RQ1}. Following a clear specification of the system, \textbf{RQ2} explores the current and future relationship with money of key stakeholders after adoption of Euro 2.0. This exploration of the stakeholders' relationship with money grounds the analysis of the impact of realised risks uncovered in \textbf{RQ3}. The severity and likelihood of realised risks to the monetary system lead to an answer to the original research question \textbf{RQ}.

\subsection{Structure of the Thesis}

\hyperref[sec:2]{Chapter 2} will give an overview of the existing body of knowledge needed to understand the remaining chapters including: traditional finance, Bitcoin, Tether, Ethereum, Estonia ID, and research methods. \hyperref[sec:3]{Chapter 3} will describe the motivation, features, and implementations of the Euro 2.0 system and prototype needed for analysis. \hyperref[sec:4]{Chapter 4} will analyse the impact of Euro 2.0 on stakeholders using design science research methods to better understand the socio-technical impact of the system. \hyperref[sec:5]{Chapter 5} will perform the full risk analysis of the Euro 2.0 implementation from the angles of Security, Scalability, and Privacy using the ISSRM model and security design patterns research methods. \hyperref[sec:6]{Chapter 6} will evaluate the findings of the risk analysis on the actual running prototype to verify the findings. And finally \hyperref[sec:7]{Chapter 7} the conclusion will answer the research question and suggest future avenues of research.

\pagebreak

%--------------------CHAPTER 2 - BRIDGE-----------------
\section{Chapter 2: Bridge of Knowledge} \label{sec:2}

\todo[inline]{Write Chapter 2}

\subsection{Traditional Finance System}

\subsection{Bitcoin}

\subsection{Tether}

\subsection{CryptoCurrencies and Payments}
Existing crypto currencies not great for payments...

\subsection{Ethereum and Smart Contracts}

\subsection{Estonia ID Card}

\subsection{Future Adoption Research}

\subsection{Security and Privacy Research}

\pagebreak

%--------------------CHAPTER 3-----------------
\section{Chapter 3} \label{sec:3}

\subsection{Introduction} \label{ssec:3.1}
This chapter introduces the Euro 2.0 digital currency system. The contents of this section is based of a prototype developed as a non profit initiative by the Euro 2.0 Foundation. All code is accessible in Github under the MIT License as indicated in \hyperref[ssec:a.1]{Appendix A.1}. The chapter answers the following research question:

\begin{quotation}
\textbf{RQ1:} How do we describe the relevant features needed for risk analysis of the Euro 2.0 digital currency?
\end{quotation}

Which is broken down into the following research questions:
\begin{quotation}
	\textbf{RQ1.1:} What are the features of the Euro 2.0 digital currency?
\end{quotation}{
\begin{quotation}
	\textbf{RQ1.2:} How are the decentralised features implemented?
\end{quotation}{
\begin{quotation}
	\textbf{RQ1.3:} How are the centralised features implemented?
\end{quotation}

Answering \textbf{RQ1} describes which features of system will be analysed in \hyperref[sec:4]{Chapters 4} and \hyperref[sec:5]{Chapter 5}. \hyperref[ssec:3.2]{Section 3.2} will briefly go through the motivation behind the Euro 2.0 digital currency system. \hyperref[ssec:3.3]{Section 3.3} will give an overview of the system architecture and features currently implemented for the Euro 2.0 answering \textbf{RQ1.1}. \hyperref[ssec:3.4]{Section 3.4} will dive into the technical implementation of the decentralised features of the system on the Ethereum platform answering \textbf{RQ1.2}. Finally \hyperref[ssec:3.5]{Section 3.5} will dive into the technical implementation of the centralised features built with traditional software development architecture, answering \textbf{RQ1.3}.

\subsection{Euro 2.0 Motivation} \label{ssec:3.2}

The idea of Euro 2.0 originated in 2014 from Kristo Käärmann's blogpost \textit{Government Backed Bitcoin}\cite{kaarmann2014government}. Following the success of Bitcoin, he saw an opportunity for a payments on a Government backed cryptocurrency, \textit{EuroCoin}, pegged one to one to real world Euro counterpart. Paypal cofounder Max Levchin supports the same idea, \textit{``I am confident that there will be a form of settlement that will be a crypto-currency''}\cite{pando2014levchin}. Bitcoin's short term fluctuations, functioning both as a commodity and a currency\todo{citation}, in value make it unreasonable for end to end holdings in a fiat based economy. Removing financial intermediaries reduces the costs of transactions.

Even some governments have expressed interests in digital currency. The Swedish central bank has expressed interest in digital currencies as a response to the move away from cash\cite{milne2016sweden}. The UK government pushed for a ``Call for Information'' on digital currencies in order to assess risks\cite{nermin2014ukcall} that was later responded by Citi bank's global Treasury and Trade Services (TTS) Technology and Innovation Team with a suggestion for government to create its own digital currency\cite{spaven2015ukcurrency}:

\begin{quotation}
	\textit{``The greatest benefits of digital currencies can be realised through the government issuing a digital form of legal tender. This currency would be less expensive, more efficient and provide greater transparency than current physical legal tender or electronic methods.''}
\end{quotation}

Following the rise in popularity of Ethereum, another blogpost by Käärmann \textit{The future of money may be in the ether}\cite{kaarmann2016ether} laid out the skeleton of the system. I can describe Euro 2.0 as solving the business problem:

\subsection{Euro 2.0 Features} \label{ssec:3.3}

\begin{quotation}
	\textit{How do I build a system to digitally store and transact fiat currency without a centralised for-profit financial intermediary conveniently, trustfully, and securely while satisfying government regulations?}
\end{quotation}

From this business problem derives the key aspects of Euro 2.0:

\begin{enumerate}
	\item Decentralised Accounts and Transactions
	\item Identity registry of all users of the system
	\item Convenient Fiat $\Longleftrightarrow$ Digital Euro exchange gateway
	\item Ability for government to issue money
	\item Features for law enforcement and AML controls
\end{enumerate}

(1) comes from the need of having impeccable record and process of storing and transacting value that no centralised party can modify in a database. This is the key benefit and usecase of a distributed ledger technology.

(2) builds trust in the system for users, merchants, and governments. If everyone is identified, any transaction can be legally ramified (assuming legislation arises supporting blockchain records in the court of law). The strong identity of Estonian ID card is a perfect fit for a digitally identifying mechanism and the main foundation of the current implementation.

(3) opens up a few possibilities of how to get users into using the digital currency. At first this exchange gateway can be a private foundation, similar to Tether\cite{tether2016whitepaper}, with a proof of reserve. Ideally this later is hosted by the central bank itself to completely remove financial intermediaries.

(4) following (3), once the central bank manages more of the system, it can become a convenient place to issue money directly into the digitial monetary system.

(5) is the most controversial aspect of the system. Bitcoin was made to sidestep the direct control of any government\cite{nakamoto2008bitcoin}. Euro 2.0 needs to be regulated from the start with full support of governments, which require means of necessary intervention that typically happen today via cooperation with traditional financial institutions.

\subsection{Euro 2.0 Decentralised Implementation} \label{ssec:3.4}

\subsection{Euro 2.0 Centralised Implementation} \label{ssec:3.5}

\pagebreak

%--------------------CHAPTER 4-----------------
\section{Chapter 4} \label{sec:4}

\todo[inline]{Write Chapter 4}

\begin{quotation}
	\textbf{RQ2:} How does adoption of Euro 2.0 digital currency impact the relationship with money for key stakeholders: users, merchants, governments, and financial institutions?
\end{quotation}

\begin{quotation}
	\textbf{rq2.1:} What is the current relationship of money for stakeholders?
\end{quotation}
\begin{quotation}
	\textbf{rq2.2:} What changes in the relationship of money for stakeholders after adopting the Euro 2.0 digital currency?
\end{quotation}
\begin{quotation}
	\textbf{rq2.3:} What are the enabling and inhibiting factors influencing the transition to the Euro 2.0 digital currency?
\end{quotation}

\pagebreak

%--------------------CHAPTER 5-----------------
\section{Chapter 5} \label{sec:5}

\todo[inline]{Write Chapter 5 \ldots}

\begin{quotation}
	\textbf{RQ3:} How do we assess the security, scalability, and privacy risks of Euro 2.0 digital currency adoption and potential impact on stakeholders?
\end{quotation}

\begin{quotation}
	\textbf{rq3.1:} What are the security, scalability, and privacy risks associated with the Euro 2.0 digital currency?
\end{quotation}
\begin{quotation}
	\textbf{rq3.2:} What is the impact of the realised risks to stakeholders?
\end{quotation}
\begin{quotation}
	\textbf{rq3.3:} What are the mitigation strategies or alternative design suggestions when implementing Euro 2.0?
\end{quotation}

\pagebreak

%--------------------CHAPTER 6 - EVALUATION-----------------
\section{Evaluation} \label{sec:6}

\todo[inline]{Write Evaluation (Chapter 6) \ldots}

\pagebreak

%-------------------------------SUMMARY---------------------------
\section{Summary} \label{sec:7}

\todo[inline]{Write Summary (Chapter 7) \ldots}

\pagebreak

%------------------------------Bibliography-----------------------------------
\addcontentsline{toc}{section}{References} \label{sec:refs}
\bibliography{thesis}{References}
\bibliographystyle{plain}
\pagebreak

%-----------------------------APPENDICES--------------------------------
\section*{Appendix A - Files} \label{sec:a}
%\label{Lisa1}
\addcontentsline{toc}{section}{Appendix A - Files}

\subsection*{A.1: Euro 2.0 Codebase} \label{ssec:a.1}
\url{http://github.com/cryptofiat}

\subsection*{A.2: Archive of Cited Web Pages}

\end{document}
